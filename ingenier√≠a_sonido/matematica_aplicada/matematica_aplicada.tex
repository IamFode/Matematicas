\documentclass[10pt]{book}
\usepackage[text=17cm,left=2.5cm,right=2.5cm, headsep=20pt, top=2.5cm, bottom = 2cm,letterpaper,showframe = false]{geometry} %configuración página
\usepackage{latexsym,amsmath,amssymb,amsfonts} %(símbolos de la AMS).7
\parindent = 0cm  %sangria
\usepackage[T1]{fontenc} %acentos en español
\usepackage[spanish]{babel} %español capitulos y secciones
\usepackage{graphicx} %gráficos y figuras.

%---------------FORMATO de letra--------------------%

\usepackage{lmodern} % tipos de letras
\usepackage{titlesec} %formato de títulos
\usepackage[backref=page]{hyperref} %hipervinculos
\usepackage{multicol} %columnas
\usepackage{tcolorbox, empheq} %cajas
\usepackage{enumerate} %indice enumerado
\usepackage{marginnote}%notas en el margen
\tcbuselibrary{skins,breakable,listings,theorems}
\usepackage[Bjornstrup]{fncychap}%diseño de portada de capitulos
\usepackage[all]{xy}%flechas
\counterwithout{footnote}{chapter}
\usepackage{xcolor}


%------------------------------------------

\newtheorem{axioma}{\large\textbf{Axioma}}[part]
\newtheorem{teo}{\textbf{TEOREMA}}[chapter]%entorno para teoremas
\newtheorem{ejem}{{\textbf{EJEMPLO}}}[chapter]%entorno para ejemplos
\newtheorem{def.}{\textbf{Definición}}[chapter]%entorno para definiciones
\newtheorem{post}{\large\textbf{Postulado}}[chapter]%entorno de postulados
\newtheorem{col.}{\textbf{Corolario}}[chapter]
\newtheorem{ej}{\textbf{Ejercicio}}[chapter]
\newtheorem{prop}{\large\textbf{Propiedad}}[part]
\newtheorem{lema}{\textbf{LEMA}}[chapter]
\newtheorem{prob}{\textbf{problema}}[chapter]

%--------------------GRÀFICOS--------------------------

\usepackage{tkz-fct}

%---------------------------------

\titleformat*{\section}{\LARGE\bfseries\rmfamily}
\titleformat*{\subsection}{\Large\bfseries\rmfamily}
\titleformat*{\subsubsection}{\large\bfseries\rmfamily}
\titleformat*{\paragraph}{\normalsize\bfseries\rmfamily}
\titleformat*{\subparagraph}{\small\bfseries\rmfamily}

\renewcommand{\thechapter}{\Roman{chapter}}
\renewcommand{\thesection}{\arabic{chapter}.\arabic{section}}
%------------------------------------------

\renewcommand{\labelenumi}{\Roman{enumi}.}%primer piso II) enumerate
\renewcommand{\labelenumii}{\arabic{enumii}$)$}%segundo piso 2)
\renewcommand{\labelenumiii}{\alph{enumiii}$)$}%tercer piso a)
\renewcommand{\labelenumiv}{$\bullet$}%cuarto piso (punto)

%----------Formato título de capítulos-------------

\usepackage{titlesec}
\renewcommand{\thechapter}{\arabic{chapter}}
\titleformat{\chapter}[display]
{\titlerule[2pt]
\vspace{4ex}\bfseries\sffamily\huge}
{\filleft\Huge\thechapter}
{2ex}
{\filleft}
\usepackage[htt]{hyphenat}


\begin{document}
\normalfont
\input xy
\xyoption{all}
\author{\Large Christian Limbert Paredes Aguilera}
\title{Matemática Aplicada }
\date{}
\pagestyle{empty}
\maketitle
\thispagestyle{empty}
\let\cleardoublepage\clearpage
\tableofcontents								%indice


%------------------------------------------
 
\let\cleardoublepage\clearpage

\chapter{Números Complejos}
\section{Definición}
%definición 1.1
\begin{tcolorbox}[colback = white]
Los números complejos $z$ se pueden definir como pares ordenados de números reales
\begin{equation}
z=(x,y)
\end{equation}
\begin{itemize}
\item Los pares $(x,0)$ se identifican como números reales.
\item Los números cmplejos de la forma $(0,y)$ se llaman números imaginarios.
\end{itemize}
\begin{equation}
Re z = x, \; Im z = y
\end{equation}
\end{tcolorbox}

%Definición 1.2
\begin{tcolorbox}[colback = white]
\begin{def.}
Dos números complejos $(x_1,y_1)$ y $x_2,y_2$ se dicen iguales si tienen iguales las partes real e imaginaria.
\begin{equation}
(x_1,y_1) = (x_2,y_2) \; \mbox{si y sólo si} \; x_1=x_2 \; e \; y_1=y_2
\end{equation}
\end{def.}
\end{tcolorbox}

\begin{tcolorbox}[colback = white]
\begin{def.}
La suma $z_1+z_2$ y el producto $z_1 \cdot z_2$ de dos números complejos $<z_1 = (x_1,y_1)$ y $z_2 = (x_2,y_2)$ se definen por las ecuaciones:
\begin{equation}
(x_1,y_1) + (x_2,y_2) = (x_1+x_2, y_1+y_2),
\end{equation}
\begin{equation}
(x_1,y_1)(x_2,y_2) = (x_1 x_2 - y_1 y_2, y_1 x_2 + x_1 y_2)
\end{equation}
\end{def.}
\end{tcolorbox}

En particular, $(x,0) + (0,y) = (x,y)$ y $(0,1)(y,0) = (0,y)$ luego 
\begin{equation}
(x,y) = (x,0) + (0,1)(y,0)
\end{equation}
Debemos notar que las ecuaciones definidas en $1.4$ y $1.5$ son las usuales cuando se restringen a los números reales.
$$(x_1,0)+(x_2,0) = (x_1+x_2,0),$$
$$(x_1,0)(x_2,0) = (x_1 x_2,0)$$  

Pensando en un número real como $x$ o como $(x,0)$, y denotando por $i$ el número imaginario puro $(0,1)$, podemos reescribir la Ecuación $1.6$ así,
\begin{equation}
(x,y)=x+iy
\end{equation}
con el convenio $z^2=zz$, $z^3=zz^2$ hallamos que $$i^2 = (0,1)(0,1) = (0\ cdot 0 - 1\cdot 1,1\cdot 0, 0 \cdot 1) =(-1,0)= -1$$
A la vista de la expresión $1.7$, las ecuaciones $1.6$ y $1.7$ se convierten en
\begin{equation}
(x_1+iy_1)+(x_2+iy_2) = (x_1+x_2)+i(y_1+y_2)
\end{equation}
\begin{equation}
(x_1+iy_2)(x_2+iy_2)=(x_1 x_2 - y_1 - y_1 y_2)+i(y_1 x_2 + x_1 +y_2)
\end{equation}

\section{propiedades algebraicas}
\begin{tcolorbox}[colback = white]
\begin{prop}[Las leyes conmutativas]
$$z_1 + z_2 = z_2 + z_1 $$  
Demostración.- \; Sea $z_1=(x_1,y_1)$ y $z_2=(x_2,y_2)$ entonces $$z_1 + z_2 = (x_1+x_2,y_1+y_2) = (x_2+x_1,y_2+y_1) = (x_2,x_1)+(y_2,y_1) = z_2+z_1$$\\
$$ z_1 z_2=z_2 z_1$$
Demostración.- \; Sea $z_1=(x_1,y_1)$ y $z_2=(x_2,y_2)$ entonces
$$z_1 z_2 = (x_1,y_1)(x_2,y_2) = (x_1x_2 - y_1y_2,y_1x_2 + x_1y_2) = (x_2x_1 - y_2y_1,y_2x_1 + x_2y_1) = (x_2,y_2)(x_1,y_1) = z_2 z_1$$\\
\end{prop}
\begin{prop}[Las leyes asociativas]
$$(z_1+z_2)+z_3 = (z_2 + z_3), \; \; (z_1 z_2)z_3 = z_1(z_2 z_3)$$
La demostración son similares a la anterior propiedad.
\end{prop}
\begin{prop}[Ley distributiva]
$$z(z_1+z_2) = zz_1 + zz_2$$
\end{prop}
\end{tcolorbox}

\begin{tcolorbox}[colback = white]
\begin{def.}[Identidad aditiva y multiplicativa]
$$z+0=z \; y \; z\cdot 1 = z$$
\end{def.}
\end{tcolorbox}

\begin{teo}[Unicidad de 0]
Supongamos que $u,v$ es una identidad aditiva, y escribimos $$(x,y)+(u,v) = (x,y)$$ donde $(x,y)$ es cualquier número complejo. Se deduce que $$x+u=x \, é \; v=y;$$ o sea $u=0$ y $v=0$. El número complejo $0=(0,0)$ es, por tanto, la única identidad aditiva. 
\end{teo}

\begin{tcolorbox}[colback = white]
\begin{def.}[Inverso aditivo]
$$-z=(-x,-y)$$ Que satisface la ecuación $z+(-z) =0$. Además, hay un sólo inverso aditivo para cada $z$, pues la ecuación $(x,y)+(u,v)=(0,0)$ implica que $u=-x$ y $v=-y$.\\
Luego si $z_1 = (x_1,y_1)$ y $z_2=(x_2,y_2)$, entonces 
$$z_1 - z_2 = (x_1-x_2,y_1-y_2) = (x_1-x_2)+i(y_1-y_2)$$
\end{def.}
\end{tcolorbox}
\end{document}
