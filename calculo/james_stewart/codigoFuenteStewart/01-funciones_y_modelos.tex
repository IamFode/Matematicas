\chapter{Funciones y modelos}
\begin{tcolorbox}[colframe=white]
    %--------------------definición 1.1.
    \begin{def.}
	Una \textbf{función} $f$ es una regla que asigna a cada elemento $x$ de un conjunto $D$ exactamente un elemento, llamado $f(x)$, de un conjunto $E$.
	$$\lbrace (x,f(x)) / x \in D \rbrace$$
	la función $f$ consta de todos los puntos $(x,y)$ en el plano coordenado tales que $y=f(x)$ y $x$ está en el dominio de $f$
    \end{def.}
\end{tcolorbox}

    %-------------------definición 1.2.
\begin{tcolorbox}[colframe=white]
    \begin{def.}
	Una función se llama \textbf{creciente} sobre un intervalo $l$ si
	\begin{center}
	    $f(x_1)<f(x_2)$ siempre que $x_1<x_2$ en $l$
	\end{center}
	Se llama \textbf{decreciente} sobre $l$ si
	\begin{center}
	    $f(x_1)>f(x_2)$ siempre que $x_1< x_2$ en $l$
	\end{center}
    \end{def.}
\end{tcolorbox}

\section{Ejercicios}
    
    \begin{enumerate}[\Large \bfseries 1.]
	
	%--------------------1.
	\item Si $f(x)=x + \sqrt{2-x}$ y $g(u)=u+\sqrt{2-u}$. ¿Es verdad que $f=g$?\\\\
	Respuesta.-\; Es verdad ya que no afecta en nada el símbolo que se podría colocar a la variable dependiente.\\\\
	
	%-------------------2.
	\item Si $f(x)=\dfrac{x_2-x}{x-1}$ y $g(x)=x$ ¿Es verdad que $f=g$?\\\\
	Respuesta.-\; No es verdad ya que el dominio de la función $g$ son todos los reales contrariamente a la función $f$ que no se cumple para $x=1$\\\\

	%-------------------3.
	\item 

    \end{enumerate}
