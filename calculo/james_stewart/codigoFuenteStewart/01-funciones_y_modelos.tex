\chapter{Funciones y modelos}
\begin{tcolorbox}[colframe=white]
    %--------------------definición 1.1.
    \begin{def.}
	Una \textbf{función} $f$ es una regla que asigna a cada elemento $x$ de un conjunto $D$ exactamente un elemento, llamado $f(x)$, de un conjunto $E$.
	$$\lbrace (x,f(x)) / x \in D \rbrace$$
	la función $f$ consta de todos los puntos $(x,y)$ en el plano coordenado tales que $y=f(x)$ y $x$ está en el dominio de $f$
    \end{def.}
\end{tcolorbox}

    %-------------------definición 1.2.
\begin{tcolorbox}[colframe=white]
    \begin{def.}
	Una función se llama \textbf{creciente} sobre un intervalo $l$ si
	\begin{center}
	    $f(x_1)<f(x_2)$ siempre que $x_1<x_2$ en $l$
	\end{center}
	Se llama \textbf{decreciente} sobre $l$ si
	\begin{center}
	    $f(x_1)>f(x_2)$ siempre que $x_1< x_2$ en $l$
	\end{center}
    \end{def.}
\end{tcolorbox}

\section{Ejercicios}
    
    \begin{enumerate}[\Large \bfseries 1.]
	
    %--------------------1.
    \item Si $f(x)=x + \sqrt{2-x}$ y $g(u)=u+\sqrt{2-u}$. ¿Es verdad que $f=g$?\\\\
    Respuesta.-\; Es verdad ya que no afecta en nada el símbolo que se podría colocar a la variable dependiente.\\\\
	
	%--------------------2.
	\item Si $f(x)=\dfrac{x^2-x}{x-1}$ y $g(x)=x$ ¿Es verdad que $f=g$?\\\\
	Respuesta.-\; No es verdad ya que el dominio de la función $g$ son todos los reales contrariamente a la función $f$ que no se cumple para $x=1$\\\\

	%--------------------3.
	\item La gráfica de una función $f$ está dada.
	    \begin{enumerate}[\bfseries (a)]
		    %----------(a)
		    \item Indique el valor de $f(1)$\\\\
		    Respuesta.-\; El valor es $3$\\\\ 

		    %----------(b)
		    \item Calcule el valor de $f(-1)$\\\\
		    Respuesta.-\; El valor es $-0.3$ aproximadamente.\\\\

		    %----------(c)
		    \item ¿Para qué valores de $x$ es $f(x)=1$?\\\\
		    Respuesta.-\;Por definición solo se cumple para $0$\\\\

		    %----------(d)
		    \item Calcule el valor de $x$ tal que $f(x)=0$\\\\
		    Respuesta.-\; El valor es aproximadamente $-0.7$\\\\ 

		    %----------(e)
		    \item Indique el dominio y el rango de $f$\\\\
		    Respuesta.-\; $f_D=\lbrace x \in f_D / -2 \leq x \leq 4 \rbrace$, y $f_R=\lbrace y \in f_R / -1 \leq y \leq 3 \rbrace$\\\\

		    %----------(f)
		    \item ¿En qué intervalo $f$ es creciente?\\\\
		    Respuesta.-\; Es creciente en el intervalo $[-2,1]$\\\\ 

	    \end{enumerate}

    %--------------------4.
    \item Las gráficas de $f$ y $g$ están dadas.

	\begin{enumerate}[\bfseries (a)]

	    %----------(a)
	    \item Indique los valores de $f(-4)$ y $g(3)$\\\\
		Respuesta.-\; $f(-4)=-2$ y $g(3)=4$\\\\

	    %----------(b)
	    \item ¿Para qué valores de $x$ es $f(x)=g(x)$?\\\\
		Respuesta.-\; Para $2,$ y $-2$\\\\ 

	    %----------(c)
	    \item Estime la solución de la ecuación $f(x)=-1$\\\\
		Respuesta.-\; $x=-3$\\\\

	    %----------(d)
	    \item ¿Sobre qué intervalo $f$ es decreciente?\\\\ 
		Respuesta.-\; Sobre $[0,4]$\\\\

	    %----------(e)
	    \item Establezca el dominio y el rango de $f$\\\\
		Respuesta.-\; El dominio de $f$ es $[-4,4]$ y el rango de $[-2,3]$\\\\

	    %----------(f)
	    \item Establezca el dominio y el rango de $g$\\\\
		Respuesta.-\; El dominio es de $[-4,3]$ y el rango de $[0.5,4]$\\\\

	\end{enumerate}

    %--------------------5.
    \item La gráfica de la figura $1$ fue registrada por un instrumento operado por el departamento de minas y geología de California en el hospital Universitario de California del Sur de Los Ángeles. Utilice esta gráfica para estimar el rango de la función aceleración vertical de suelo, en la Universidad de California del Sur, durante el terremoto de Northridge.\\\\
	Respuesta.-\; El rango de la función de aceleración  vertical del suelo es dado por el intervalo $[-1,3]$\\\\ 

    %--------------------6.
    \item En esta sección se discuten ejemplos de funciones cotidianas: la población es una función del tiempo, el costo, de envío postal es una función del peso, la temperatura del agua es una función del tiempo. De otros tres ejemplos de funciones de la vida cotidiana que se describen verbalmente. ¿Qué puede decir sobre el dominio y el rango de cada una de sus funciones? Si es posible, trace una gráfica de cada función.\\\\
	Respuesta.-\; 

    %--------------------7.
    \item No cumple con la definición de función.\\\\

    %--------------------8.
    \item Cumple con la definición de función por lo tanto $f_D=\lbrace x / -2 \leq x \leq 2 \rbrace$ y $f_R=\lbrace y \in f(x) / -1 \leq y \leq 2 \rbrace$\\\\

    %--------------------9.
    \item Cumple con la definición de función por lo tanto $f_D=\lbrace x / -3 \leq x < -2 \; \cup \; -2 \leq x \leq 2 \rbrace$ y $f_R=\lbrace y \in f(x) / -3 \leq x < -2 \; \cup \; -2 \leq y \leq 2.6 \rbrace$\\\\

    %--------------------10.
    \item No cumple con la definición de función por lo tanto no se tiene un dominio y rango.\\\\

    %--------------------11.
    \item En la figura se muestra una gráfica de la temperatura media global $T$ durante el siglo $XX$. Estime lo siguiente.
	\begin{enumerate}[\bfseries (a)]

	    %----------(a)
	    \item La temperatura media mundial en $1950$\\\\
		Respuesta.-\; 14.5 grados centígrados.\\\\

	    %----------(b)
	    \item El año en que la temperatura promedio fue de $14.2$ C.\\\\
		Respuesta.-\; Aproximadamente en $1905$\\\\

	    %----------(c)
	    \item ¿En qué año la temperatura fue más baja? ¿Más alta?\\\\
		Respuesta.-\; Fue más baja en $1920$ y más alta en $2010$\\\\

	    %----------(d)
	    \item El rango de $T$\\\\
		Respuesta.-\; El rango se encuentra en el intervalo de $[12.8,14.8]$\\\\ 

	\end{enumerate}

    %--------------------12.
    \item Los arboles crecen más rápido y forman anillos más amplios en los años cálidos y crecen más lentamente y forman anillos más angostos en los años más fríos. La figura muestra anillos anchos del pino Siberiano de $1500$ a $2000$.

	\begin{enumerate}[\bfseries (a)]

	    %----------(a)
	    \item ¿Cuál es el rango de la función de ancho de anillo?\\\\
		Respuesta.-\; El rango es de $0.1$ a $1.41$ aproximadamente.\\\\

	    %----------(b)
	    \item ¿Qué dice la gráfica acerca de la temperatura de la tierra? ¿La gráfica refleja las erupciones volcánicas de la mitad del siglo $XIX$?\\\\
		Respuesta.-\; La temperatura cada vez es más cálida mientras pasa los años. \\
		Se ve un pequeño realce entre $1800-1890$ donde podría haber existido algunas erupciones volcánicas.\\\\

	\end{enumerate}

    %--------------------13.
    \item Se ponen unos cubitos de hielo en un vaso, se llena el vaso con agua fría y luego se coloca sobre una mesa. Describa cómo cambia la temperatura del agua conforme transcurre el tiempo. Luego trace una gráfica de la temperatura del agua como una función del tiempo transcurrido.\\\\
    Respuesta.-\; La temperatura va disminuyendo constantemente en función a la temperatura ambiente y del tiempo.

            \begin{center}
                \begin{tikzpicture}[scale=0.9, draw opacity = .6]
                    % abscisa y ordenada
                    \tkzInit[xmax= 3,xmin=-2,ymax=3,ymin=-2]
                    \tiny\tkzLabelXY[opacity=0.6,step=1, orig=false]
                    % label x, f(x)
                    \tkzDrawX[opacity= .6,label=x,right=0.3]
                    \tkzDrawY[opacity= .6,label=f(x),below = -0.6]
                    %dominio y función
                    \draw [domain=-2:3,thick] plot(\x,{\x}); 
                \end{tikzpicture}
            \end{center}  

    %--------------------14.
    \item Tres corredores compiten en una carrera de $100$ metros. La gráfica muestra la distancia recorrida como una función del tiempo de cada corredor. Describa en palabras lo que la gráfica indica acerca de esta carrera. ¿Quién ganó la carrera? ¿Cada corredor terminó la carrera?\\\\
    Respuesta.-\; Gano la carrera el competidor $A$. Efectivamente cada competidor acabo la carrera.\\\\ 

    %--------------------15.
    \item La gráfica muestra el consumo de energía para un día de septiembre en San Francisco. ($P$ se mide en megawatts: $t$ se registra en horas a partir de la medianoche).

	\begin{enumerate}[\bfseries (a)]

	    %----------(a)
	    \item ¿Cual fue el consumo de potencia a las $6:00$? ¿A las $18:00$?\\\\
	    Respuesta.-\; Para las $6:00$ el consumo de potencia es $500$ megawatts. y para las $18:00$ es $720$ megawatts.\\\\

	    %----------(b)
	    \item ¿Cuándo fue el consumo de energía más bajo? ¿Cuándo fue el más alto? ¿Estos tiempos parecen razonables?\\\\
	    Respuesta.-\; El más bajo fue a hrs. $4:00$ y el mas alto fue en a hrs. $12:00$. Los tiempos son razonables por la actividad que se puede realizar a esas horas.\\\\

	\end{enumerate}

    %--------------------16.
    \item Trace una gráfica aproximada del número de horas de luz en función de la época del año.\\\\
    Respuesta.-\;

    \end{enumerate}
