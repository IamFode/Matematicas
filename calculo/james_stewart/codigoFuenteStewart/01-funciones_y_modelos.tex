\chapter{Funciones y modelos}
\begin{tcolorbox}[colframe=white]
    %--------------------definición 1.1.
    \begin{def.}
	Una \textbf{función} $f$ es una regla que asigna a cada elemento $x$ de un conjunto $D$ exactamente un elemento, llamado $f(x)$, de un conjunto $E$.
	$$\lbrace (x,f(x)) / x \in D \rbrace$$
	la función $f$ consta de todos los puntos $(x,y)$ en el plano coordenado tales que $y=f(x)$ y $x$ está en el dominio de $f$
    \end{def.}
\end{tcolorbox}

    %-------------------definición 1.2.
\begin{tcolorbox}[colframe=white]
    \begin{def.}
	Una función se llama \textbf{creciente} sobre un intervalo $l$ si
	\begin{center}
	    $f(x_1)<f(x_2)$ siempre que $x_1<x_2$ en $l$
	\end{center}
	Se llama \textbf{decreciente} sobre $l$ si
	\begin{center}
	    $f(x_1)>f(x_2)$ siempre que $x_1< x_2$ en $l$
	\end{center}
    \end{def.}
\end{tcolorbox}

\section{Ejercicios}
    
    \begin{enumerate}[\Large \bfseries 1.]
	
    %--------------------1.
    \item Si $f(x)=x + \sqrt{2-x}$ y $g(u)=u+\sqrt{2-u}$. ¿Es verdad que $f=g$?\\\\
    Respuesta.-\; Es verdad ya que no afecta en nada el símbolo que se podría colocar a la variable dependiente.\\\\
	
	%--------------------2.
	\item Si $f(x)=\dfrac{x^2-x}{x-1}$ y $g(x)=x$ ¿Es verdad que $f=g$?\\\\
	Respuesta.-\; No es verdad ya que el dominio de la función $g$ son todos los reales contrariamente a la función $f$ que no se cumple para $x=1$\\\\

	%--------------------3.
	\item La gráfica de una función $f$ está dada.
	    \begin{enumerate}[\bfseries (a)]
		    %----------(a)
		    \item Indique el valor de $f(1)$\\\\
		    Respuesta.-\; El valor es $3$\\\\ 

		    %----------(b)
		    \item Calcule el valor de $f(-1)$\\\\
		    Respuesta.-\; El valor es $-0.3$ aproximadamente.\\\\

		    %----------(c)
		    \item ¿Para qué valores de $x$ es $f(x)=1$?\\\\
		    Respuesta.-\;Por definición solo se cumple para $0$\\\\

		    %----------(d)
		    \item Calcule el valor de $x$ tal que $f(x)=0$\\\\
		    Respuesta.-\; El valor es aproximadamente $-0.7$\\\\ 

		    %----------(e)
		    \item Indique el dominio y el rango de $f$\\\\
		    Respuesta.-\; $f_D=\lbrace x \in f_D / -2 \leq x \leq 4 \rbrace$, y $f_R=\lbrace y \in f_R / -1 \leq y \leq 3 \rbrace$\\\\

		    %----------(f)
		    \item ¿En qué intervalo $f$ es creciente?\\\\
		    Respuesta.-\; Es creciente en el intervalo $[-2,1]$\\\\ 

	    \end{enumerate}

    %--------------------4.
    \item Las gráficas de $f$ y $g$ están dadas.

	\begin{enumerate}[\bfseries (a)]

	    %----------(a)
	    \item Indique los valores de $f(-4)$ y $g(3)$\\\\
		Respuesta.-\; $f(-4)=-2$ y $g(3)=4$\\\\

	    %----------(b)
	    \item ¿Para qué valores de $x$ es $f(x)=g(x)$?\\\\
		Respuesta.-\; Para $2,$ y $-2$\\\\ 

	    %----------(c)
	    \item Estime la solución de la ecuación $f(x)=-1$\\\\
		Respuesta.-\; $x=-3$\\\\

	    %----------(d)
	    \item ¿Sobre qué intervalo $f$ es decreciente?\\\\ 
		Respuesta.-\; Sobre $[0,4]$\\\\

	    %----------(e)
	    \item Establezca el dominio y el rango de $f$\\\\
		Respuesta.-\; El dominio de $f$ es $[-4,4]$ y el rango de $[-2,3]$\\\\

	    %----------(f)
	    \item Establezca el dominio y el rango de $g$\\\\
		Respuesta.-\; El dominio es de $[-4,3]$ y el rango de $[0.5,4]$\\\\

	\end{enumerate}

    %--------------------5.
    \item La gráfica de la figura $1$ fue registrada por un instrumento operado por el departamento de minas y geología de California en el hospital Universitario de California del Sur de Los Ángeles. Utilice esta gráfica para estimar el rango de la función aceleración vertical de suelo, en la Universidad de California del Sur, durante el terremoto de Northridge.\\\\
	Respuesta.-\; El rango de la función de aceleración  vertical del suelo es dado por el intervalo $[-1,3]$\\\\ 

    %--------------------6.
    \item En esta sección se discuten ejemplos de funciones cotidianas: la población es una función del tiempo, el costo, de envío postal es una función del peso, la temperatura del agua es una función del tiempo. De otros tres ejemplos de funciones de la vida cotidiana que se describen verbalmente. ¿Qué puede decir sobre el dominio y el rango de cada una de sus funciones? Si es posible, trace una gráfica de cada función.\\\\
	Respuesta.-\; 
    \end{enumerate}
