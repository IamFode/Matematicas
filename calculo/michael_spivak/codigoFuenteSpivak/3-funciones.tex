\chapter{Funciones}


    %definición 1.1
\begin{tcolorbox}[colframe=white]
    \begin{def.}
	El conjunto de los números a los cuales se aplica una función recibe el nombre de \textbf{dominio} de la función.
    \end{def.}
\end{tcolorbox}

    %definición 1.2
\begin{tcolorbox}[colframe=white]
    \begin{def.}
	Si $f$ \; y \; $g$ son dos funciones cualesquiera, podemos definir una nueva función $f+g$ denominada \textbf{suma} de $f+g$ mediante la ecuación:
	$$(f+g)(x)=f(x)+g(x)$$
	Para el conjunto de todos los $x$ que están a la vez en el dominio de $f$ y en el dominio de $g$, es decir: $$dominio \; (f+g)=dominio \; f \; \cap \; dominio \; g$$\\
    \end{def.}
\end{tcolorbox}

    %definición 1.3
\begin{tcolorbox}[colframe=white]
    \begin{def.}
	El dominio de $f \cdot g$ es $dominio \; f \cap \; dominio \; g$ $$(f \cdot g)(x)=f(x)\cdot g(x)$$\\
    \end{def.}
\end{tcolorbox}

    %definición 1.4 
\begin{tcolorbox}[colframe=white]
    \begin{def.}
	Se expresa por dominio $f$ $\cap$ dominio $g$ $\cap$ $\lbrace x:g(x)\neq 0 \rbrace$
	$$\left( \dfrac{f}{g}\right) (x)=\dfrac{f(x)}{g(x)}$$ \\
    \end{def.}
\end{tcolorbox}

%definición 1.5
\begin{tcolorbox}[colframe=white]
    \begin{def.}[Función constante]
	$$(c \cdot g)(x)=c \cdot g(x)$$\\
    \end{def.}
\end{tcolorbox}

%teorema 1.1
\begin{teo}
    $(f+g)+h=f+(g+h)$\\\\
    Demostración.- \; \textbf{La demostración es característica de casi todas las demostraciones que prueban que dos funciones son iguales: se debe hacer ver que las dos funciones tienen el mismo dominio y el mismo valor para cualquier número del dominio.} Obsérvese que al interpretar la definición de cada lado se obtiene:
    \begin{center}
	\begin{tabular}{r c l}
	    $\left[ (f+g) + h \right](x)$&=&$(f+g)(x)+h(x)$\\
	    &=&$\left[ f(x) +g(x) \right] +h(x)$\\\\
	    &y&\\\\
	    $\left[ f+(g+h) \right](x)$&=&$f(x)+(g+h)(x)$\\
	    &=&$f(x)+\left[ g(x)+h(x) \right]$\\\\
	\end{tabular}
    \end{center}
    Es esta demostración no se ha mencionado la igualdad de los dos dominios porque esta igualdad parece obvia desde el momento en que empezamos a escribir estas ecuaciones: el dominio de $(f+g)+h$ y el de $f+(g+h)$ es evidentemente dominio $f$ $\cap$ dominio $g$ $\cap$ dominio $h$. Nosotros escribimos, naturalmente $f+g+h$ por $(f+g)+h=f+(g+h)$\\\\
\end{teo}

%teorema 1.2
\begin{teo}
    Es igual fácil demostrar que $(f\cdot g)\cdot g=f\cdot (g \cdot h)$ y ésta función se designa por $f \cdot g \cdot h$. Las ecuaciones $f+g=g+f$ \; y \; $f\cdot g=g \cdot f$ no deben presentar ninguna dificultad.\\\\
\end{teo}

%definición 1.6
\begin{tcolorbox}[colframe=white]
    \begin{def.}[Composición de función]
	$$(f \circ g)(x)=f(g(x))$$
	El dominio de $f\circ g$ es $\lbrace $ $x$ : $x$ está en el dominio de $g$ \: y \; $g(x)$ está en el dominio de $f$ $\rbrace$
	$$D_{f \circ g}= \lbrace x \; / \; x \in D_g \; \land \; g(x)\in D_f \rbrace$$
    \end{def.}
\end{tcolorbox}

    %propiedad 1.1
\begin{tcolorbox}[colframe=white]
    \begin{prop}
    $(f \circ g) \circ h = f \circ (g \circ h)$   La demostración es una trivalidad.
    \end{prop}
\end{tcolorbox}

%definición  1.7 
\begin{tcolorbox}
    \begin{def.} 
	Una \textbf{función} es una colección de pares de números con la siguiente propiedad: Si $(a,b)$ \; y \; $(a,c)$ pertenecen ambos a la colección, entonces $b=c$; en otras palabras, la colección no debe contener dos pares distintos con el mismo primer elemento.\\
    \end{def.}
\end{tcolorbox}

%definición 1.8
\begin{tcolorbox}
    \begin{def.} 
	Si $f$ es una función, el \textbf{dominio} de $f$ es el conjunto de todos los $a$ para los que existe algún $b$ tal que $(a,b)$ está en $f$. Si $a$ está en el dominio de $f$, se sigue de la definición de función que existe, en efecto, un número $b$ único tal que $(a,b)$ está en $f$. Este $b$ único se designa por $f(a)$.\\  
    \end{def.}
\end{tcolorbox}


\section{Problemas}

    \begin{enumerate}[\Large \bfseries 1.]

	%--------------------1.
	\item Sea $f(x)=1/(1+x)$. Interpretar lo siguiente:
	    \begin{enumerate}[\bfseries (i)]

		%----------(i)
		\item $f(f(x))$ (¿Para que $x$ tiene sentido?)\\\\
		Respuesta.- \; Sea $f\left( \dfrac{1}{1+x} \right)$ entonces $\dfrac{1}{1 + \dfrac{1}{1+x}}$, por lo tanto $\dfrac{1-x}{x+2}$ de donde llegamos a la conclusión de que $x$ se cumple para todo número real de $1$ y $-2$\\\\

		%----------(ii)
		\item $f\left( \dfrac{1}{x} \right)$\\\\
		Respuesta.- \; $\dfrac{1}{1 + \dfrac{1}{x}}=\dfrac{1}{\dfrac{x+1}{x}}=\dfrac{x}{x+1}$ por lo tanto se cumple para todo $x\neq -1, 0$\\\\

		%----------(iii)
		\item $f(cx)$\\\\
		Respuesta.- \; $\dfrac{1}{1+cx}$ donde se cumple para todo $x\neq -1$ si $c\neq 0$\\\\

		%----------(iv)
		\item $f(x+y)$\\\\
		Respuesta.- \; $\dfrac{1}{1+x+y}$ donde se cumple para todo $x+y\neq-1$\\\\

		%----------(v)
		\item $f(x) + f(y)$\\\\
		Respuesta.- \; $\dfrac{1}{1+x} + \dfrac{1}{1+y}=\dfrac{x+y+2}{(1+x)(1+y)}$ siempre y cuando $x\neq-1$ y $y \neq -1$\\\\

		%----------(vi)
		\item ¿Para que números $c$ existe un número $x$ tal que $f(cx)=f(x)$?\\\\
		Respuesta.- \; Para todo $c$ ya que $f(c\cdot 0)=f(0)$\\\\

		%----------(vii)
		\item ¿Para que números $c$ se cumple que $f(cx)=f(x)$ para dos números distintos $x$?\\\\
		Respuesta.- \;  Solamente $c=1$ ya que $f(x)=f(cx)$ implica que $x=cx$, y esto debe cumplirse por lo menos para un $x \geq 1$\\\\

	    \end{enumerate}

	%--------------------2.  
	\item Sea $g(x)=x^2$ y sea 
	\begin{equation*}
	    h(x) = \left\lbrace
		\begin{array}{rl}
		    0, & x \; racional\\
		    1, & x \; irracional
		\end{array}
	    \right.
	\end{equation*}

	\begin{enumerate}[\bfseries (i)]

	    %----------(i)
	    \item ¿Para cuáles $y$ es $h(y) \leq y$?\\\\
	    Respuesta-. \; Se cumple para $y\geq 0$ si $y$ es racional, o para todo $y\geq 1$\\\\

	    %----------(ii)
	    \item ¿Para cuáles $y$ es $h(y) \leq g(y)$?\\\\
	    Respuesta-. \; Para $-1\leq y \leq 1$ siempre que $y$ sea racional y para todo $y$ tal que $|y|\leq1$\\\\

	    %----------(iii)
	    \item ¿Qué es $g(h(z)) - h(z)$?\\\\
	    Respuesta-. \; 
	    \begin{equation*}
		g(h(z)) = \left\lbrace
		    \begin{array}{rl}
			0, & z^2 \; racional\\
			1, & z^2 \; irracional
		    \end{array}
		\right.
	    \end{equation*}

	    Por lo tanto el resultado es $0$\\\\

	    %----------(iv)
	    \item ¿Para cuáles $w$ es $g(w)\leq w $?\\\\
	    Respuesta-. \; Para todo $w$ tal que $0\leq w\leq 1$\\\\

	    %----------(v)
	    \item ¿Para cuáles $\epsilon$ es $g(g(\epsilon)) = g(\epsilon)$?\\\\
	    Respuesta-. \; Para $-1,0,1$\\\\ 

	\end{enumerate}

	%--------------------3.
	\item Encontrar el dominio de las funciones definidas por las siguientes fórmulas:
	    \begin{enumerate}[\bfseries (i)]

	    %----------(i)
	    \item $f(x)=\sqrt{1-x^2}$\\\\
	    Respuesta.- \; Por la propiedad de raíz cuadrada, se tiene  $1-x^2 \geq 0$ entonces $x^2 \leq 1$ por lo tanto el dominio son todos los $x$ tal que $|x| \leq 1$\\\\

	    %-----------(ii)
	    \item $f(x)=\sqrt{1-\sqrt{1-x^2}}$\\\\
	    Respuesta.- \; Se observa claramente que el dominio es $-1\leq x \leq 1$\\\\

	    %-----------(iii)
	    \item $f(x)=\dfrac{1}{x-1} + \dfrac{1}{x-2}$\\\\
	    Respuesta.- \; Operando un poco tenemos $$f(x) = \dfrac{2x-3}{(x-1)(x-2)},$$ sabemos que el denominador no puede ser $0$ por lo tanto el $D_{f} = \lbrace x\; / \; x \neq 1, \; x\neq  2 \rbrace$\\\\ 
    
	    %-----------(iv)
	    \item $f(x)=\sqrt{1-x^2} + \sqrt{x^2-1}$\\\\
	    Respuesta.- \; Claramente notamos que el dominio de $f$ son $-1$ y $1$ ya que si se toma otros números daría un número imaginario.\\\\

	    %-----------(v)
	    \item $f(x)=\sqrt{1-x}+\sqrt{x-2}$\\\\
	    Respuesta.- \; Notamos que no se cumple para ningún $x$ ya que si $0\leq x \leq 1$ entonces no se cumple para $\sqrt{x-2} $ y si $x\geq 2$ no se cumple para $\sqrt{1-x}$\\\\ 

	    \end{enumerate}

	%--------------------4.
	\item Sean $S(x)=x^2,$ $P(x)=2^x$ y $s(x)=sen x$. Determinar los siguientes valores. En cada caso la solución debe ser un número.\\\\

	\begin{enumerate}[\bfseries (i)]

	    %----------(i)
	    \item $(S \circ P)(y)$\\\\
	    Respuesta.- \; Por definición se tiene que $(S \circ P)(y)=S(P(y))$ entonces $S(2^y)=2^{2y}$ siempre y cuando $D_{S \circ P}=\lbrace y / y \in D_P \land P(y) \in D_S\rbrace$\\\\
	    
	    %----------(ii)
	    \item $(S \circ s)(y)$\\\\
	    Respuesta.- \; Por definición tenemos que $(S \circ s)(y)=S(s(y))$ entonces $S(\sen y)=\sen^2 y$ siempre y cuando $D_{S \circ s}=\lbrace y / y \in D_s \land S(y) \in D_S \rbrace$\\\\

	    %----------(iii)
	    \item $(S \circ P \circ s)(t)+(s \circ P)(t)$\\\\
	    Respuesta.- \; $(S \circ P \circ s)(t)+(s \circ P)(t) = S((P \circ s)(t))+s(P(t)) = S(P(s(t))) + s(P(t))=S(P(\sen t)) + s(2^t)=S(2^{\sen t}) +  \sen 2^t = 2^{2 \sen t} +\sen 2^t$\\\\  

	    %----------(iv)
	    \item $s(t^3)$\\\\
	    Respuesta.- \; $s(t^3)=\sen t^3$\\\\

	\end{enumerate}

	    %-------------------5.
	\item Expresar cada una de las siguientes funciones en términos de $S,P,s$ usando solamente $+,\cdot , \circ$\\\\
	\begin{enumerate}[\bfseries (i)]
	    
	    %----------(i)
	    \item $f(x)=2^{\sen x}$\\\\
	    Respuesta.- \; Claramente vemos que $P \circ s$\\\\ 

	    %----------(ii)
	    \item $f(x) = \sen 2^x$\\\\
	    Respuesta.- \; $s \circ P$\\\\

	    %----------(iii)
	    \item $f(x) = \sen x^2$\\\\
	    Respuesta.- \; $s \circ S$\\\\

	    %----------(iv)
	    \item $f(x) = \sen x$\\\\
	    Respuesta.- \; $S \circ s$\\\\

	    %----------(v)
	    \item $f(t) = 2^{2t}$\\\\
	    Respuesta.- \; $P \circ P$\\\\

	    %----------(vi)
	    \item $f(u)=\sen (2^u + 2^{u^2})$\\\\
	    Respuesta.- \; $s \circ (P + P \circ S)$\\\\

	    %----------(vii)
	    \item $f(y) = \sen (\sen (\sen (2^{2^{2^{\sen y}}})))$\\\\
	    Respuesta.- \; $s \circ s \circ s \circ P \circ P \circ P \circ s$\\\\

	    %----------(viii)
	    \item $f(a)= 2^{\sen^2 a} +  \sen(a^2) + 2^{sen(a^2 + \sen a)}$\\\\
	    Respuesta.- \; $P \circ S \circ s  + s \circ S + P \circ s \circ (S + s)$\\\\

	\end{enumerate}

	%--------------------6.
	\item 
	\begin{enumerate}[\bfseries (a)]
	    \item Si $x_1, ... , x_n$ son números distintos, encontrar una función polinómica $f_i$ de grado $n-1$ que tome el valor $1$ en $x_i$ y $0$ en $x_j$ para $j \neq i.$ Indicación: El producto de todos los $(x-x_j)$ para $j \neq i$ es $0$ en $x_j$ si $j \neq i.$ Este producto es designado generalmente por $$\prod\limits_{j=1_{j \neq i}}^n (x-x_j)$$ donde el símbolo $\prod$ (pi mayúscula) desempeña para productos el mismo papel que $\sum$ para sumas.\\\\

	\end{enumerate}

	\end{enumerate}
