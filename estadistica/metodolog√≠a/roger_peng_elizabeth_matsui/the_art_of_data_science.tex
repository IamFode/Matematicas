\documentclass[10pt]{book}
\usepackage[text=17cm,left=4cm,right=4cm, headsep=20pt, top=2.5cm, bottom = 2cm,letterpaper,showframe = false]{geometry} %configuración página
\usepackage{latexsym,amsmath,amssymb,amsfonts} %(símbolos de la AMS).7
\parindent = 0cm  %sangria
\usepackage[T1]{fontenc} %acentos en español
\usepackage[spanish]{babel} %español capitulos y secciones
\usepackage{graphicx} %gráficos y figuras.

%---------------FORMATO de letra--------------------%

\usepackage{lmodern} % tipos de letras
\usepackage{titlesec} %formato de títulos
\usepackage[backref=page]{hyperref} %hipervinculos
\usepackage{multicol} %columnas
\usepackage{tcolorbox, empheq} %cajas
\usepackage{enumerate} %indice enumerado
\usepackage{marginnote}%notas en el margen
\tcbuselibrary{skins,breakable,listings,theorems}
\usepackage[Bjornstrup]{fncychap}%diseño de portada de capitulos
\usepackage[all]{xy}%flechas
\counterwithout{footnote}{chapter}
\usepackage{xcolor}
\usepackage[htt]{hyphenat}
%--------------------GRÀFICOS--------------------------

\usepackage{tkz-fct}

%---------------------------------

\titleformat*{\section}{\LARGE\bfseries\rmfamily}
\titleformat*{\subsection}{\Large\bfseries\rmfamily}
\titleformat*{\subsubsection}{\large\bfseries\rmfamily}
\titleformat*{\paragraph}{\normalsize\bfseries\rmfamily}
\titleformat*{\subparagraph}{\small\bfseries\rmfamily}

%------------------------------------------

\renewcommand{\labelenumi}{\Roman{enumi}.}%primer piso II) enumerate
\renewcommand{\labelenumii}{\arabic{enumii}$)$}%segundo piso 2)
\renewcommand{\labelenumiii}{\alph{enumiii}$)$}%tercer piso a)
\renewcommand{\labelenumiv}{$\bullet$}%cuarto piso (punto)

%----------Formato título de capítulos-------------

\usepackage{titlesec}
\renewcommand{\thechapter}{\arabic{chapter}}
\titleformat{\chapter}[display]
{\titlerule[2pt]
\vspace{4ex}\bfseries\sffamily\huge}
{\filleft\Huge\thechapter}
{2ex}
{\filleft}

\begin{document}

\normalfont
\input xy
\xyoption{all}
\author{\Large Apuntes por FODE}
\title{\small Roger D. Peng y Elizabeth Matsui \\ \vspace{1cm} \large The art of data science}
\date{}
\pagestyle{empty}
\maketitle
\thispagestyle{empty}
\let\cleardoublepage\clearpage
\tableofcontents								%indice


%------------------------------------------
 
\let\cleardoublepage\clearpage
\chapter{Data Analysis es arte}
Imagina que le preguntas a un compositor cómo escribe sus canciones. Hay muchas herramientas a las que puede recurrir. Tenemos una comprensión general de cómo debe estructurarse una buena canción: cuánto tiempo debe ser, cuántos versos, tal vez haya un verso seguido de un coro, etc. En otras palabras, existe un marco abstracto para las canciones en general. De manera similar, tenemos la teoría musical que nos dice que ciertas combinaciones de notas y acordes funcionan bien juntas y otras combinaciones no suenan bien. Por muy buenas que puedan ser estas herramientas, en última instancia, el conocimiento de la estructura de la canción y la teoría musical por sí solo no es una buena canción. Se necesita algo más.\\
\textbf{Todo es arte, por ende es importante darse cuenta de que el análisis de datos es un arte.}\\
Los analistas de datos tienen muchas herramientas a su disposición, desde regresión lineal hasta árboles de clasificación e incluso aprendizaje profundo, y todas estas herramientas se han enseñado cuidadosamente a las computadoras. Pero, en última instancia, un analista de datos debe encontrar una manera de reunir todas las herramientas y aplicarlas a los datos para responder una pregunta relevante, una pregunta de interés para las personas.\\
En 1991, Daryl Pregibon, un destacado estadístico anteriormente de AT\& T Research y ahora de Google, dijo en referencia al proceso de análisis de datos2 que $“$los estadísticos tienen un proceso que adoptan pero que no comprenden completamente$”$.\\
\textbf{Lo que nos hemos propuesto hacer en este libro es escribir el proceso de análisis de datos. Lo que describimos no es una "fórmula" específica para el análisis de datos, algo como "aplicar este método y luego ejecutar esa prueba", sino que es un proceso general que se puede aplicar en una variedad de situaciones.}\\
\chapter{Epiciclos del análisis}
\textbf{En realidad, el análisis de datos es un proceso altamente iterativo y no lineal, mejor reflejado por una serie de epiciclos, en los cuales se aprende información en cada paso, que luego informa si (y cómo) refinar y rehacer, el paso que se acaba de realizar, o si (y cómo) continuar con el siguiente paso. Un epiciclo es un círculo pequeño cuyo centro se mueve alrededor de la circunferencia de un círculo más grande.}\\
\begin{enumerate}[\bfseries 1.]
\item Las expectativas se desarrollan.
\item Recolectar Datos.
\item expectativas matemáticas con datos
\end{enumerate}
Estos $"$pasos$"$ se engranan con este otro epiciclo siguiente
\begin{enumerate}[\bfseries 1.-]
\item Planteando la pregunta.
\item Análisis exploratorio de datos.
\item Análisis exploratorio de datos.
\item Interpretar.
\item Comunicar.
\end{enumerate}
\section{Preparando la escena}
Dado que un análisis de datos supone que los datos ya se han recopilado, incluye el desarrollo y el refinamiento de una pregunta y el proceso de análisis e interpretación de los datos. Es importante señalar que, aunque un análisis de datos a menudo se realiza sin realizar un estudio, también se puede realizar como un componente de un estudio.\\
\section{epiciclo de análisis}
Hay 5 actividades centrales del análisis de datos:
\begin{enumerate}[\bfseries 1.]
\item Formular y refinar la pregunta.
\item Explorar los datos.
\item Construir modelos estadísticos formales.
\item Interpretar los resultados.
\item Comunicar los resultados.
\end{enumerate} 
Estas 5 actividades pueden ocurrir en diferentes escalas de tiempo: por ejemplo, puede pasar por los 5 en el transcurso de un día, pero también tratar con cada uno, para un proyecto grande, en el transcurso de muchos meses.\\\\
Para cada una de las cinco actividades principales, es fundamental que participe en los siguientes pasos:
\begin{enumerate}[\bfseries a)]
\item Establecer expectativas.
\item Recopilar información (datos), comparar los datos con sus expectativas y si las expectativas no coinciden.
\item Revisar sus expectativas o corregir los datos para que sus datos y sus expectativas coincidan. 
\end{enumerate}
La iteración a través de este proceso de 3 pasos es lo que llamamos el $"$epiciclo del análisis de datos$"$.\\

\begin{center}
\begin{tabular}{|m{2.6cm}| m{3cm} | m{3cm}| m{3cm} | }
\hline
&\textbf{Establecer expectativas}&\textbf{Recopilar información}&\textbf{Revisar expectativas}\\
\hline
\textbf{Pregunta}&pregunta de interés para la audiencia&búsqueda de literatura / expertos&agudizar la pregunta\\
\hline
\textbf{Explorar datos}&los datos son apropiados para preguntas&hacer gráficos exploratorios de datos&refinar la pregunta o llamar más datos\\
\hline
\textbf{Modelo formal}&modelo primario responde a la pregunta&ajustar modelos secundarios, análisis de sensibilidad&revisar el modelo formal para incluir más predictores\\
\hline
\textbf{Interpretación}&La interpretación de los análisis proporciona una respuesta específica y significativa a la pregunta.&interpretar la totalidad de los análisis centrándose en los tamaños del efecto y la incertidumbre&revisar EDA y / o modelos para proporcionar una respuesta específica e interpretable\\
\hline
\textbf{Comunicación}&El proceso y los resultados del análisis son entendidos, completos y significativos para la audiencia.&Buscar retroalimentación&revisar análisis o enfoque de presentación\\
\hline
\end{tabular}
\end{center}

\section{Estableciendo expectativas}
Desarrollar expectativas es el proceso de pensar deliberadamente en lo que espera antes de hacer algo, como inspeccionar sus datos, realizar un procedimiento o ingresar un comando. Por ejemplo averiguar el costo de una comida en un restaurant de lujo puede ser una expectativa.

\section{Recopilando información}
Los resultados de esa operación son los datos que necesita recopilar y luego determina si los datos que recopiló coinciden con sus expectativas. Para extender la metáfora del restaurante, cuando vas al restaurante, obtener el cheque es recopilar los datos.

\section{Comparación de expectativas con datos}
Un indicador clave de qué tan bien va su análisis de datos es lo fácil o difícil que es hacer coincidir los datos que recopiló con sus expectativas originales.

\section{Aplicación del proceso de Epicyle of Analysis}
Antes de analizar un par de ejemplos, repasemos los tres pasos que se deben utilizar para cada actividad de análisis de datos básicos. Estos son: 
\begin{enumerate}[\bfseries 1.]
\item Establecer expectativas.
\item Recopilar información (datos), comparar los datos con sus expectativas y, si las expectativas no coinciden.
\item Revisar sus expectativas o corregir los datos para que sus expectativas y los datos coincidan.
\end{enumerate}
Los modelos estadísticos sirven para producir una formulación precisa de su pregunta para que pueda ver exactamente cómo desea usar sus datos, ya sea para estimar un parámetro específico o para hacer una predicción.\\
\textbf{argumentaríamos que un buen análisis de datos requiere comunicación, retroalimentación y luego acciones en respuesta. Su análisis de datos trajo preguntas adicionales al frente, ya que esta es una característica de un análisis de datos exitoso.}

\chapter{Formular y perfeccionar la pregunta}
\textbf{Hacer análisis de datos requiere pensar bastante y creemos que cuando ha completado un buen análisis de datos, ha pasado más tiempo pensando que haciendo.}

\section{Tipos de preguntas}
Los seis tipos de preguntas son:
\begin{enumerate}[\bfseries 1.]
\item Descriptivo.
\item Exploratorio.
\item Inferencial.
\item Predictivo.
\item Causal.
\item Mecanismo.
\end{enumerate}
\textbf{Una pregunta descriptiva} es aquella que busca resumir una característica de un conjunto de datos. Los ejemplos incluyen determinar la proporción de hombres, el número medio de porciones de frutas y verduras frescas por día o la frecuencia de enfermedades virales en un conjunto de datos recopilados de un grupo de personas. No hay interpretación del resultado en sí, ya que el resultado es un hecho, un atributo del conjunto de datos con el que está trabajando. \\
\textbf{Una pregunta exploratoria} es aquella en la que analiza los datos para ver si existen patrones, tendencias o relaciones entre las variables. Estos tipos de análisis también se denominan análisis de $“$generación de hipótesis$”$ porque en lugar de probar una hipótesis como se haría con una pregunta inferencial, causal o mecanicista, se buscan patrones que respalden la propuesta de una hipótesis. Si tuviera la idea general de que la dieta estaba relacionada de alguna manera con enfermedades virales, podría explorar esta idea examinando las relaciones entre una variedad de factores dietéticos y enfermedades virales. Usted encuentra en su análisis exploratorio que los individuos que consumían una dieta alta en ciertos alimentos tenían menos enfermedades virales que aquellos cuya dieta no estaba enriquecida con estos alimentos, por lo que propone la hipótesis de que entre los adultos, comer al menos 5 porciones al día de fruta fresca y las verduras se asocia con menos enfermedades virales por año.\\
\textbf{Una pregunta inferencial} sería una reafirmación de esta hipótesis propuesta como una pregunta y se respondería analizando un conjunto diferente de datos, que en este ejemplo, es una muestra representativa de adultos en los EE. UU. Al analizar este conjunto diferente de datos, ambos están determinando si la asociación que observó en su análisis exploratorio se mantiene en una muestra diferente y si se mantiene en una muestra que es representativa de la población adulta de EE. UU., Lo que sugeriría que la asociación es aplicable a todos los adultos en los Estados Unidos. En otras palabras, podrá inferir lo que es cierto, en promedio, para la población adulta en los EE. UU. A partir del análisis que realice en la muestra representativa.\\
\textbf{Una pregunta predictiva} sería aquella en la que se pregunta qué tipos de personas consumirán una dieta rica en frutas y verduras frescas durante el próximo año. En este tipo de preguntas, usted está menos interesado en lo que hace que alguien coma una dieta determinada, solo en lo que predice si alguien comerá esta dieta determinada. Por ejemplo, un ingreso más alto puede ser uno de los últimos factores de predicción, y es posible que no sepa (o ni siquiera le importe) por qué las personas con ingresos más altos tienen más probabilidades de comer una dieta rica en frutas y verduras frescas, pero lo más importante es que los ingresos son un factor que predice este comportamiento. Aunque una pregunta inferencial podría decirnos que las personas que consumen cierto tipo de alimentos tienden a tener menos enfermedades virales, la respuesta a esta pregunta no nos dice si comer estos alimentos provoca una reducción en el número de enfermedades virales, que sería la caso de una pregunta causal.\\
\textbf{Una pregunta causal} se refiere a si cambiar un factor cambiará otro factor, en promedio, en una población. A veces, el diseño subyacente de la recopilación de datos, de forma predeterminada, permite que la pregunta que hace sea causal. Un ejemplo de esto serían los datos recopilados en el contexto de un ensayo aleatorizado, en el que las personas fueron asignadas al azar a comer una dieta rica en frutas y verduras frescas o una que era bajo en frutas y verduras frescas. En otros casos, incluso si sus datos no son de un ensayo aleatorio, puede adoptar un enfoque analítico diseñado para responder una pregunta causal.\\
Finalmente, ninguna de las preguntas descritas hasta ahora conducirá a una respuesta que nos diga, si la dieta, efectivamente, causa una reducción en el número de enfermedades virales, cómo la dieta conduce a una reducción en el número de enfermedades virales. Una pregunta que pregunta cómo una dieta rica en frutas y verduras frescas conduce a una reducción en el número de enfermedades virales sería \textbf{una pregunta mecanicista}.\\
si un análisis de datos tiene como objetivo responder una pregunta inferencial, las preguntas descriptivas y exploratorias también deben responderse durante el proceso de respuesta a la pregunta inferencial. Para continuar con nuestro ejemplo de dieta y enfermedades virales, no saltaría directamente a un modelo estadístico de la relación entre una dieta alta en frutas y verduras frescas y el número de enfermedades virales sin haber determinado la frecuencia de este tipo de dieta y enfermedades virales. y su relación entre sí en esta muestra. Un segundo punto es que el tipo de pregunta que hace está determinado en parte por los datos disponibles (a menos que planee realizar un estudio y recopilar los datos necesarios para realizar el análisis). Por ejemplo, es posible que desee hacer una pregunta causal sobre la dieta y las enfermedades virales para saber si una dieta rica en frutas y verduras frescas provoca una disminución en el número de enfermedades virales, y el mejor tipo de datos para responder a esta pregunta causal es una en la que las dietas de las personas cambian de una rica en frutas y verduras frescas a una que no lo es, o viceversa. Si este tipo de conjunto de datos no existe, lo mejor que puede hacer es aplicar métodos de análisis a los datos de observación o, en cambio, responder a una pregunta inferencial sobre la dieta y las enfermedades virales.

\section{Aplicar el epiciclo para formular y perfeccionar su pregunta}
Ahora puede usar la información sobre los tipos de preguntas y las características de las buenas preguntas como guía para refinar su pregunta. Para lograr esto, puede iterar a través de los 3 pasos de:
\begin{enumerate}[\bfseries 1.]
\item Establecer expectativas.
\item Recopilar información (datos), comparar los datos con sus expectativas y, si las expectativas no coinciden.
\item Revisar sus expectativas o corregir los datos para que sus expectativas y los datos coincidan.
\end{enumerate}
Los modelos estadísticos sirven para producir una formulación precisa de su pregunta para que pueda ver exactamente cómo desea usar sus datos, ya sea para estimar un parámetro específico o para hacer una predicción.\\
\textbf{argumentaríamos que un buen análisis de datos requiere comunicación, retroalimentación y luego acciones en respuesta. Su análisis de datos trajo preguntas adicionales al frente, ya que esta es una característica de un análisis de datos exitoso.}

\section{Caracteristicas de una buena pregunta}
Para empezar, la pregunta debe ser de interés para su audiencia, cuya identidad dependerá del contexto y el entorno en el que esté trabajando con los datos. Si está en el mundo académico, la audiencia puede ser sus colaboradores, la comunidad científica, los reguladores gubernamentales, sus patrocinadores de Establecimiento y perfeccionamiento de la Pregunta 21 y / o el público. Si está trabajando en una startup, su audiencia es su jefe, el liderazgo de la empresa y los inversores.\\
Puede asegurarse de que su pregunta se basa en un marco plausible utilizando su propio conocimiento del área temática y haciendo un poco de investigación, que juntos pueden ser de gran ayuda en términos de ayudarlo a resolver si su pregunta se basa en un marco plausible .\\
La especificidad también es una característica importante de una buena pregunta. Un ejemplo de una pregunta general es: ¿Es mejor para usted llevar una dieta más saludable? Trabajar hacia la especificidad refinará su pregunta e informará directamente qué pasos tomar cuando comience a buscar datos. El proceso de aumento de la especificidad debería conducir a una pregunta final y refinada como: $"$¿Comer al menos 5 porciones al día de frutas y verduras frescas provoca menos infecciones del tracto respiratorio superior (resfriados)?$"$\\

\section{Traducir una pregunta en un problema de datos}
A medida que refina su pregunta, dedique algún tiempo a identificar los posibles factores de confusión y a pensar si su conjunto de datos incluye información sobre estos posibles factores de confusión.\\
Otro tipo de problema que puede ocurrir cuando se utilizan datos inapropiados es que el resultado no es interpretable porque la forma subyacente en la que se recopilaron los datos conduce a un resultado sesgado.\\\\
Las dos tareas principales que se debe abordar son: 
\begin{enumerate}[\bfseries 1.]
\item pensar en cómo su pregunta cumple o no con las características de una buena pregunta y 
\item determinar qué tipo de pregunta está haciendo para que tenga una buena idea. 
\end{enumerate}
buena comprensión de qué tipo de conclusiones se pueden (y no se pueden) sacar una vez finalizado el análisis de datos.\\

\chapter{Análisis exploratorio de datos}
El análisis de datos exploratorio más confiable consiste en visualizar datos utilizando una representación gráfica de los datos.\\
Hay varios objetivos del análisis de datos exploratorios, que son: 
\begin{enumerate}[\bfseries 1.]
    \item Para determinar si hay algún problema con su conjunto de datos. 
    \item Para determinar si la pregunta que está haciendo puede ser respondida por los datos que tiene.
    \item Desarrollar un bosquejo de la respuesta a su pregunta.
\end{enumerate}
\textbf{Explorará los datos para determinar si hay problemas con el conjunto de datos y para determinar si puede responder a su pregunta con este conjunto de datos.}\\
Es importante notar que aquí, nuevamente, se aplica el concepto de epiciclo de análisis. Debe tener una expectativa de cómo se verá su conjunto de datos y si su pregunta puede ser respondida por los datos que tiene. Si el contenido y la estructura del conjunto de datos no coinciden con sus expectativas, entonces deberá volver atrás y averiguar si sus expectativas eran correctas (pero hubo un problema con los datos) o, alternativamente, sus expectativas eran incorrectas, por lo que no puede usar el conjunto de datos para responder la pregunta y necesitará encontrar otro conjunto de datos. También debe tener alguna expectativa de cuáles serán los niveles de ozono, así como si el ozono de una región debe ser más alto (o más bajo) que el de otra.\\

    \section{Lista de verificación de análisis de datos exploratorios: un estudio de caso}
    En esta sección repasaremos una “lista de verificación” informal de cosas que hacer al embarcarse en un análisis de datos exploratorio. Como ejemplo continuo, usaré un conjunto de datos sobre los niveles de ozono por hora en los Estados Unidos para el año 2014. Los elementos de la lista de verificación son:
    \begin{enumerate}[\bfseries 1.]
        \item Formule su pregunta
        \item Lea sus datos. 
        \item Verifique el empaquetado. 
        \item Mire la parte superior e inferior de sus datos. 
        \item Verifique sus $“$n$”$ s. 
        \item Valide con al menos una fuente de datos externa. 
        \item Haga una gráfica. 
        \item Pruebe primero la solución fácil.
        \item Haga un seguimiento.
    \end{enumerate}

        \subsection{Formule su pregunta}
        En particular, una pregunta o hipótesis aguda puede servir como una herramienta de reducción de dimensión que puede eliminar variables que no son inmediatamente relevantes para la pregunta. \\
        \textbf{ Por lo general, es una buena idea dedicar unos minutos a averiguar cuál es la pregunta que realmente le interesa y reducirla para que sea lo más específica posible}\\
        una de las preguntas más importantes que puede responder con un análisis exploratorio de datos es $"$¿Tengo los datos correctos para responder esta pregunta?$"$ A menudo, esta pregunta es difícil de responder al principio, pero puede volverse más clara a medida que revisamos y examinamos los datos.

        \subsection{Leer en sus datos}
        ¿Alguna vez recibió un regalo antes del momento en que se le permitió abrirlo? Seguro, todos lo hemos hecho. El problema es que el presente está envuelto, pero deseas desesperadamente saber qué hay dentro. ¿Qué debe hacer una persona en esas circunstancias? Bueno, puede agitar un poco la caja, tal vez golpearla con los nudillos para ver si hace un sonido hueco, o incluso pesarla para ver qué tan pesado es. Así es como debe pensar en su conjunto de datos antes de comenzar a analizarlo de verdad. \\
        Más importante aún, puede examinar las clases de cada una de las columnas para asegurarse de que estén correctamente especificadas (es decir, las letras numéricas son numéricas y las cadenas de caracteres, etc.)

        \subsection{Mire la parte superior e inferior de sus datos}
        A menudo, es útil mirar el $"$principio$"$ y el $"$final$"$ de un conjunto de datos inmediatamente después de comprobar el paquete. Esto le permite saber si los datos se leyeron correctamente, si las cosas están formateadas correctamente y si todo está ahí. Si sus datos son datos de series de tiempo, asegúrese de que las fechas al principio y al final del conjunto de datos coincidan con lo que espera que sean el período inicial y final.\\

        \subsection{ABC: Siempre revise sus $"$n$"$ s}
        En general, contar cosas suele ser una buena forma de averiguar si algo está mal o no. En el caso más simple, si espera que haya 1,000 observaciones y resulta que solo hay 20, sabe que algo debe haber salido mal en alguna parte. Pero hay otras áreas que puede verificar según su aplicación.

        \subsection{Validar con al menos una fuente de datos externa}
        Es muy importante asegurarse de que sus datos coincidan con algo fuera del conjunto de datos. Le permite asegurarse de que las mediciones estén aproximadamente en línea con lo que deberían ser y sirve como una verificación de qué otras cosas podrían estar mal en su conjunto de datos.\\

        \subsection{Haga un gráfica}
        Hacer un diagrama para visualizar sus datos es una buena manera de comprender mejor su pregunta y sus datos. El trazado puede ocurrir en diferentes etapas de un análisis de datos. Para el trazado puede ocurrir en la fase exploratoria o más adelante en la fase de presentación / comunicación. Hay dos razones clave para realizar un gráfico de sus datos. Están creando expectativas y comprobando las desviaciones de las expectativas. En las primeras etapas del análisis, puede estar equipado con una pregunta / hipótesis, pero es posible que tenga poca idea de lo que está sucediendo en los datos. Es posible que haya echado un vistazo a algunos de ellos para hacer algunas comprobaciones de cordura, pero si su conjunto de datos es lo suficientemente grande, será difícil simplemente mirar todos los datos. Entonces, hacer algún tipo de gráfico, que sirva como resumen, será una herramienta útil para establecer expectativas sobre cómo deberían verse los datos. Una vez que tenga una buena comprensión de los datos, una buena pregunta / hipótesis y un conjunto de expectativas sobre lo que los datos deberían decir en relación con su pregunta, hacer un gráfico puede ser una herramienta útil para ver qué tan bien los datos coinciden con sus expectativas. . Los gráficos son particularmente buenos para permitirle ver desviaciones de lo que podría esperar. Por lo general, las tablas son buenas para resumir datos al presentar elementos como medias, medianas u otras estadísticas. Los gráficos, sin embargo, pueden mostrarle esas cosas, así como mostrarle cosas que están lejos de la media o la mediana, para que pueda verificar si se supone que algo está tan lejos. A menudo, lo que es obvio en una trama se puede ocultar en una tabla.\\

        \subsection{Pruebe primero la solución fácil}
        Es importante destacar que si no encuentra evidencia de una señal en los datos usando solo una gráfica o análisis simple, entonces a menudo es poco probable que encuentre algo usando un análisis más sofisticado.\\

            \subsubsection{Pon a prueba tu solución}
            Siempre debe pensar en formas de desafiar los resultados, especialmente si esos resultados concuerdan con sus expectativas anteriores. Recuerde que anteriormente notamos que tres estados tenían algunos valores inusualmente altos de ozono. No sabemos si estos valores son reales o no (por ahora, supongamos que son reales), pero podría ser interesante ver si el mismo patrón de este / oeste se mantiene si eliminamos estos estados que tienen actividad inusual.\\

        \subsection{Preguntas de seguimiento}
        En este punto, es útil considerar algunas preguntas de seguimiento.
        \begin{enumerate}[ \bfseries 1.]
            \item \textbf{¿Tienes los datos correctos?} A veces, al final de un análisis de datos exploratorio, la conclusión es que el conjunto de datos no es realmente apropiado para esta pregunta de Análisis de datos exploratorios.
            \item \textbf{¿Necesitas otros datos?} Si bien los datos parecían adecuados para responder la pregunta planteada, vale la pena señalar que el conjunto de datos solo cubrió un año (2014). Puede valer la pena examinar si el patrón este / oeste se mantiene durante otros años, en cuyo caso tendríamos que salir y obtener otros datos.
            \item \textbf{¿Tienes la pregunta correcta?} En este caso, no está claro que la pregunta que intentamos responder tenga relevancia inmediata, y los datos realmente no indicaron nada para aumentar la relevancia de la pregunta.
        \end{enumerate}
        \textbf{El objetivo del análisis exploratorio de datos es hacer que piense en sus datos y razone sobre su pregunta. En este punto, podemos refinar nuestra pregunta o recopilar nuevos datos, todo en un proceso iterativo para llegar a la verdad}

\chapter{Uso de modelos para explorar sus datos}
La construcción de modelos, como todo el proceso de análisis de datos en sí, es un proceso iterativo. Los modelos se utilizan para proporcionar reducción de datos y para darle una idea de la población sobre la que está tratando de hacer inferencias. Es importante establecer primero sus expectativas sobre cómo un modelo debe caracterizar un conjunto de datos antes de aplicar un modelo a los datos. Luego, puede verificar si su modelo se ajusta a sus expectativas. A menudo, habrá características del conjunto de datos que no se ajustan a su modelo y tendrá que refinar su modelo o examinar el proceso de recopilación de datos.
 
\chapter{Lo primero, inferencia}
En general, el objetivo de la inferencia es poder hacer una declaración sobre algo que no se observa e idealmente poder caracterizar cualquier incertidumbre que tenga sobre esa declaración. La inferencia es difícil debido a la diferencia entre lo que puede observar y lo que en última instancia desea saber.

    \section{Identificar la población}
    Identificar la población es la tarea más importante. Si no puede identificar o describir coherentemente la población, entonces no puede hacer una inferencia. Solo para. Una vez que haya averiguado cuál es la población y sobre qué característica de la población desea hacer una declaración (por ejemplo, la media), luego puede traducir eso en una declaración más específica utilizando una estadística formal.\\ 
    \section{Describe el proceso de muestreo}
    ¿Cómo llegaron los datos de la población a su computadora? Ser capaz de describir este proceso es importante para determinar si los datos son útiles para hacer inferencias sobre las características de la población. Describir el proceso de muestreo depende de su capacidad para describir coherentemente la población.
    \section{Describe un modelo para la población}
    Necesitamos tener una representación abstracta de cómo los elementos de la población se relacionan entre sí. Por lo general, esto viene en forma de un modelo estadístico que podemos representar usando notación matemática. Sin embargo, en situaciones más complejas, podemos recurrir a representaciones algorítmicas que no se pueden escribir claramente en papel (muchos enfoques de aprendizaje automático deben describirse de esta manera).
    \textbf{no debe obsesionarse con desarrollar un modelo correcto; en su lugar, debe identificar un modelo que le sea útil y que cuente una historia sobre los datos y los procesos subyacentes que está tratando de estudiar.}\\
    Las tres cosas que debemos hacer para hacer una inferencia son:
    \begin{enumerate}[\bfseries 1.]
	\item \textbf{Define la población.} 
	\item \textbf{Describe el proceso de muestro.}
	\item \textbf{Describe un modelo para la población.}
    \end{enumerate}
    \section{Factores que afectan la calidad de la inferencia}
    Los factores clave que afectan la calidad de una inferencia que podría hacer se relacionan con violaciones en nuestro pensamiento sobre el proceso de muestreo y el modelo para la población.\\
    si no podemos definir coherentemente la población, entonces cualquier inferencia que hagamos a la población se definirá de manera similar de manera vaga.
    \textbf{Este fenómeno a veces se denomina sesgo de selección porque las cantidades que estima están sesgadas hacia la selección de la población que muestreó.}
    \section{Las poblaciones se presentan en muchas formas}
	\subsection{Series de tiempo}
	Independientemente de lo que elija, es importante dejar claro a qué población se refiere antes de intentar hacer inferencias a partir de los datos.
	\subsection{Procesos naturales}
	es posible que tengamos datos de que solo se mide en el espacio. Por ejemplo, podemos tener un mapa de los epicentros de todos los terremotos que han ocurrido en un área. Entonces, ¿cuál es la población? Un enfoque común es decir que existe un proceso estocástico no observado que arroja terremotos al azar en el área y que nuestros datos representan una muestra aleatoria de este proceso. En ese caso, estamos utilizando los datos para intentar obtener más información sobre este proceso no observado.
	\subsection{Datos como población}	
	Una técnica que siempre es posible, pero que no se usa comúnmente, es tratar el conjunto de datos como una población. En este caso, no hay inferencia porque no hay muestreo. Debido a que su conjunto de datos es la población, no hay incertidumbre sobre ninguna característica de la población. Puede que esto no suene como una estrategia útil, pero hay circunstancias en las que se puede utilizar para responder preguntas importantes. En particular, hay momentos en los que no nos preocupan las cosas fuera del conjunto de datos. Por ejemplo, es común en las organizaciones analizar los datos salariales para asegurarse de que a las mujeres no se les paga menos que a los hombres por un trabajo comparable o que no existen grandes desequilibrios entre empleados de diferentes grupos étnicos. En este escenario, las diferencias en los salarios entre los diferentes grupos se pueden calcular en el conjunto de datos y se puede ver si las diferencias son lo suficientemente grandes como para ser motivo de preocupación. La cuestión es que los datos responden directamente a una pregunta de interés.

\chapter{Modelado formal}
Escribir un modelo estadístico usando notación matemática, a diferencia del lenguaje natural, lo obliga a ser preciso en su descripción del modelo y en su declaración de lo que está tratando de lograr, como estimar un parámetro.
    \section{¿Cuáles son los objetivos del modelado formal?}
    Un objetivo clave del modelado formal es desarrollar una especificación precisa de su pregunta y cómo se pueden utilizar sus datos para responder a esa pregunta. Los modelos formales le permiten identificar claramente lo que está tratando de inferir a partir de los datos y qué forma adoptan las relaciones entre las características de la población.
    \section{Marco general}
    \begin{enumerate}[\bfseries 1.]
	\item \textbf{Establecer expectativas} Establecer expectativas viene en forma de desarrollar un modelo primario que representa su mejor sentido de lo que proporciona la respuesta a su pregunta. Este modelo se elige en función de la información que tenga actualmente disponible. 
	\item \textbf{Recopilación de información} Una vez que se establece el modelo primario, querremos crear un conjunto de modelos secundarios que desafíen al modelo primario de alguna manera. 
	\item \textbf{Revisión de expectativas} Si nuestros modelos secundarios tienen éxito en desafiar nuestro modelo primario y ponen en duda las conclusiones del modelo primario, entonces es posible que debamos ajustar o modificar el modelo primario para reflejar mejor lo que hemos aprendido de los modelos secundarios.
    \end{enumerate}
	\subsection{Modelo primario}
	A menudo es útil comenzar con un modelo principal. Es probable que este modelo se derive de cualquier análisis exploratorio que ya haya realizado y servirá como candidato principal para algo que resuma sucintamente sus resultados y coincida con sus expectativas. Es importante darse cuenta de que en un momento dado de un análisis de datos, el modelo principal no es necesariamente el modelo final. Es simplemente el modelo con el que comparará otros modelos secundarios. El proceso de comparar su modelo con otros modelos secundarios a menudo se conoce como análisis de sensibilidad, porque le interesa ver qué tan sensible es su modelo a los cambios, como agregar o eliminar predictores o eliminar valores atípicos en los datos.
	\subsection{Modelos secundarios}
	Una vez que se haya decidido por un modelo principal, normalmente desarrollará una serie de modelos secundarios. El propósito de estos modelos es probar la legitimidad y solidez de su modelo principal y potencialmente generar evidencia contra su modelo principal. Si los modelos secundarios logran generar evidencia que refuta las conclusiones de su modelo primario, es posible que deba volver a examinar el modelo primario y determinar si sus conclusiones siguen siendo razonables.
    \section{Análisis asociativos}
    Los análisis asociativos son aquellos en los que buscamos una asociación entre dos o más características en presencia de otros factores potencialmente confusos. Hay tres clases de variables en las que es importante pensar en un análisis asociativo:
    \begin{enumerate}[\bfseries 1.]
	\item \textbf{Resultado} El resultado es la característica de su conjunto de datos que se cree que cambiará junto con su predictor clave. Incluso si no está haciendo una pregunta causal o mecanicista, por lo que no cree necesariamente que el resultado responda a los cambios en el predictor clave, todavía es necesario definir un resultado para la mayoría de los enfoques formales de modelado.
	\item \textbf{Predictor clave} Queremos saber cómo cambia el resultado con este predictor clave.
	\item \textbf{Posibles factores de confusión} Ésta es una gran clase de predictores que están relacionados tanto con el predictor clave como con el resultado. Es importante comprender bien qué son y si están disponibles en su conjunto de datos.
    \end{enumerate}
    Una vez que haya identificado estas tres clases de variables en su conjunto de datos, puede comenzar a pensar en el modelado formal en un entorno asociativo.\\
    A menudo es preferible elegir el modelo que sea más simple. Hay dos razones para esto. Primero, con un modelo más simple puede ser más fácil contar una historia sobre lo que está sucediendo en los datos a través de los diversos parámetros del modelo. Por ejemplo, es más fácil explicar una tendencia lineal que explicar una tendencia exponencial. En segundo lugar, los modelos más simples, desde una perspectiva estadística, son más “eficientes”, por lo que hacen un mejor uso de los datos por parámetro que se está estimando.
    \section{Análisis de predicción}
    Los análisis de predicción a menudo dejarán al algoritmo de predicción determinar la importancia de cada predictor y determinar la forma funcional del modelo. Para muchos análisis de predicción, no es posible escribir literalmente el modelo que se está utilizando para predecir porque no se puede representar utilizando la notación matemática estándar. Muchas rutinas modernas de predicción están estructuradas como algoritmos o procedimientos que toman entradas y las transforman en salidas. El camino que toman las entradas para transformarse en salidas puede ser muy no lineal y los predictores pueden interactuar con otros predictores en el camino. Normalmente, no hay parámetros de interés que tratemos de estimar; de hecho, muchos procedimientos algorítmicos no tienen ningún parámetro estimable en absoluto. La clave para recordar con los análisis de predicción es que generalmente no nos importan los detalles específicos del modelo.
	\subsection{Expectativas}
	¿Cuál es el escenario ideal en un problema de predicción? Generalmente, lo que queremos es un predictor, o un conjunto de predictores, para producir una buena separación en el resultado. El objetivo de la mayoría de los problemas de predicción es identificar un conjunto de predictores que minimice el tamaño.
	\subsection{Evaluación}
	Para problemas de predicción, decidir el siguiente paso después del ajuste inicial del modelo puede depender de algunos factores.
	\begin{enumerate}[\bfseries 1.]
	    \item \textbf{Calidad de la predicción} ¿La precisión del modelo es lo suficientemente buena para sus propósitos? Esto depende del objetivo final y los riesgos asociados con las acciones posteriores.
	    \item \textbf{Ajuste de modelo} Un sello distintivo de los algoritmos de predicción son sus numerosos parámetros de ajuste. A veces, estos parámetros pueden tener grandes efectos en la calidad de la predicción si se modifican, por lo que es importante estar informado del impacto de los parámetros de ajuste para cualquier algoritmo que utilice. No existe un algoritmo de predicción para el que un solo conjunto de parámetros de ajuste funcione bien para todos los problemas.
	    \item \textbf{Disponibilidad de otros datos} Muchos algoritmos de predicción son bastante buenos para explorar la estructura de conjuntos de datos grandes y complejos e identificar una estructura que pueda predecir mejor su resultado. Si encuentra que su modelo no está funcionando bien, incluso después de algunos ajustes de los parámetros de ajuste, es probable que necesite datos adicionales para mejorar su predicción.
	\end{enumerate}
    \section{Resumen}
    El modelado formal es típicamente el aspecto más técnico del análisis de datos, y su propósito es establecer con precisión cuál es el objetivo del análisis y proporcionar un marco riguroso para desafiar sus hallazgos y probar sus suposiciones. El enfoque que adopte puede variar dependiendo principalmente de si su pregunta se trata fundamentalmente de estimar una asociación que desarrolla una buena predicción.

\chapter{Inferencia frente a predicción: implicaciones para la estrategia de modelado}
Comprender si está respondiendo una pregunta inferencial o una pregunta de predicción es un concepto importante porque el tipo de pregunta que está respondiendo puede influir en gran medida en la estrategia de modelado que siga. Las cosas clave para recordar son:
\begin{enumerate}[\bfseries 1.]
    \item \textbf{Para las preguntas inferencial,} el objetivo es típicamente estimar una asociación entre un predictor de interés y el resultado. Por lo general, solo hay un puñado de predictores de interés (o incluso solo uno), sin embargo, generalmente hay muchas variables de confusión potenciales a considerar. \textbf{El objetivo clave del modelado es estimar una asociación y, al mismo tiempo, asegurarse de que se ajusta adecuadamente a los posibles factores de confusión}. A menudo, se realizan análisis de sensibilidad para ver si las asociaciones de interés son sólidas para diferentes conjuntos de factores de confusión.
    \item \textbf{Para las preguntas de predicción,} el objetivo es identificar un modelo que prediga mejor el resultado. \textbf{El objetivo clave es desarrollar un modelo con buena habilidad de predicción y estimar una tasa de error razonable a partir de los datos.}
\end{enumerate}
    \section{Resumen}
    En cualquier análisis de datos, debe preguntarse a sí mismo $"$¿Estoy haciendo una pregunta inferencial o una pregunta de predicción?$"$ Esto debe aclararse antes de analizar cualquier dato, ya que la respuesta a la pregunta puede guiar toda la estrategia de modelado. Enmarcar correctamente la pregunta y aplicar la estrategia de modelado adecuada, puede desempeñar un papel importante en los tipos de conclusiones que extrae de los datos.
\chapter{Interpretando tus resultados}
    \section{Principios de interpretación}
    \begin{enumerate}[\bfseries 1.]
	\item Revise su pregunta original. La naturaleza del resultado incluye tres características: su direccionalidad, magnitud e incertidumbre. La incertidumbre es una evaluación de la probabilidad de que el resultado se haya obtenido por casualidad.
	\item Empiece con el modelo estadístico primario para orientarse y centrarse en la naturaleza del resultado en lugar de en una evaluación binaria del resultado.
	\item Desarrolle una interpretación general basada en:
	    \begin{enumerate}[\bfseries (a)]
		\item La totalidad de su análisis.
		\item El contexto de lo que se conoce sobre el tema.
	    \end{enumerate}
	\item Considere las implicaciones, que lo guiarán a la hora de determinar qué acción (es), si corresponde, se debe tomar como resultado de la respuesta a su pregunta. Es importante señalar que el epiciclo del análisis también se aplica a la interpretación.
    \end{enumerate}
    Aunque puede estar en uno de los últimos pasos del análisis de datos cuando está interpretando formalmente sus resultados, es posible que deba volver al análisis de datos exploratorios o al modelado para hacer coincidir las expectativas con los datos.
	\subsection{repase su pregunta}
	Esto puede parecer una declaración frívola, pero no es raro que las personas se pierdan a medida que avanzan en el proceso de análisis exploratorio y modelado formal. Esto suele ocurrir cuando un analista de datos se desvía demasiado de su curso en busca de un hallazgo incidental que aparece en el proceso de análisis exploratorio de datos o modelado formal. Luego, el modelo o modelos finales proporcionan una respuesta a otra pregunta que surgió durante los análisis en lugar de la pregunta original. \textbf{Recuerde que el sesgo es un problema sistemático con la recopilación o el análisis de los datos que da como resultado una respuesta incorrecta a su pregunta.} La manera de poder encontrar algún sesgo a una pregunta es suponer que está sesgada. \textbf{El punto es que hacer una pausa para realizar un experimento mental deliberado sobre las fuentes de sesgo es de vital importancia, ya que en realidad es la única forma de evaluar el potencial de un resultado sesgado. Este experimento mental también debe realizarse cuando está formulando y refinando su pregunta y también cuando está realizando análisis exploratorios y modelado.}
	\subsection{Comience con el modelo primario y evalúe la direccionalidad, la magnitud y la incertidumbre del resultado.}
	El segundo principio es comenzar con un solo modelo y enfocarse en el continuo completo del resultado, incluida su direccionalidad y magnitud, y el grado de certeza (o incertidumbre) que existe sobre si el resultado de la muestra que analizó refleja el resultado real. No pierda mucho tiempo preocupándose por qué modelo individual empezar, porque al final considerará todos sus resultados y este ejercicio de interpretación inicial sirve para orientarlo y proporcionar un marco para su interpretación final.
	    \subsubsection{Direccionalidad}
	    ¿La direccionalidad positiva del resultado coincide con sus expectativas que se han desarrollado a partir del análisis de datos exploratorios? Si es así, está en buena forma y puede pasar a la siguiente actividad de interpretación. Si no, hay un par de posibles explicaciones. Primero, es posible que sus expectativas no sean correctas porque el análisis exploratorio se realizó incorrectamente o su interpretación de los análisis exploratorios no fue correcta. En segundo lugar, el análisis exploratorio y su interpretación del mismo pueden ser correctos, pero el modelado formal puede haberse realizado incorrectamente.
	    \subsubsection{Magnitud}
	    una parte clave de la interpretación de la magnitud del resultado es comprender cómo se compara la magnitud del resultado con lo que sabe acerca de este tipo de información en la población que le interesa.
	    \subsubsection{Incertidumbre}
	    Recuerde que su modelo se ha construido para ajustarse a los datos recopilados de una muestra de la población en general. Para evaluar si el resultado de la muestra es simplemente "ruido" aleatorio, utilizamos medidas de incertidumbre. La probabilidad de que su muestra refleje la respuesta para la población general varía dependiendo de qué tan cerca (o lejos) esté el resultado de su muestra del resultado verdadero para la población general. \\
	    Una herramienta que proporciona una medida de incertidumbre más continua es el \textbf{intervalo de confianza.} Un intervalo de confianza es un rango de valores que contiene el resultado de su muestra y usted tiene cierta confianza en que también contiene el resultado verdadero para la población general. Es importante darse cuenta de que debido a que el intervalo de confianza se construye a partir de los datos, el intervalo en sí es aleatorio. Por lo tanto, si tuviéramos que recopilar nuevos datos, el intervalo que construiríamos sería ligeramente diferente. Sin embargo, la verdad, es decir, el valor poblacional del parámetro, siempre permanecería igual. \\
	    Otra herramienta para medir la incertidumbre es, por supuesto, el valor p, que simplemente es la probabilidad de obtener el resultado de la muestra de 0,28 kg / m2 (o más extremo) cuando la verdadera relación entre el consumo de refrescos no dietéticos y el IMC en la población general es 0.  Centrarse principalmente en el valor p es un enfoque arriesgado para interpretar la incertidumbre porque puede llevar a ignorar información más importante necesaria para una interpretación cuidadosa y precisa de sus resultados. 
	\subsection{Desarrolle una interpretación general considerando la totalidad de sus análisis e información externa}
	Ahora que ha dedicado una buena cantidad de esfuerzo a interpretar los resultados de su modelo primario, el siguiente paso es desarrollar una interpretación general de sus resultados considerando tanto la totalidad de sus análisis como la información externa a sus análisis.  La interpretación de los resultados de su modelo primario sirve para establecer la expectativa de su interpretación general cuando considera todos sus análisis. \textbf{No existe un solo modelo que por sí solo proporcione la respuesta a su pregunta. En cambio, existen modelos adicionales que sirven para cuestionar el resultado obtenido en el modelo primario.} Un tipo común de modelo secundario es el modelo que se construye para determinar qué tan sensibles son los resultados en su modelo primario a los cambios en los datos. Un ejemplo clásico es la eliminación de valores atípicos para evaluar el grado en que cambia el resultado del modelo primario. Un segundo ejemplo es evaluar el efecto de posibles factores de confusión en los resultados del modelo primario. Aunque el modelo primario ya debería contener factores de confusión clave, normalmente existen otros posibles factores de confusión que deberían evaluarse. \\
	¿cómo interpreta cómo estos resultados del modelo secundario afectan su resultado primario? Puede recurrir al paradigma de: direccionalidad, magnitud e incertidumbre.\\
	La información externa es tanto el conocimiento general que usted o los miembros de su equipo tienen sobre el tema, como los resultados de análisis similares e información sobre la población objetivo. 
    \section{Trascendencia}
    Ahora que ha interpretado sus resultados y tiene conclusiones en la mano, querrá pensar en las implicaciones de sus conclusiones. Después de todo, el objetivo de hacer un análisis suele ser informar una decisión o emprender una acción. A veces las implicaciones son sencillas, pero otras veces las implicaciones requieren un poco de reflexión.
\chapter{Comunicación}
    Nos centraremos en:
    \begin{enumerate}[\bfseries 1.]
	\item Cómo utilizar la comunicación de rutina como una de las herramientas necesarias para realizar un buen análisis de datos.
	\item Cómo transmitir los puntos clave de su análisis de datos cuando se comunica de manera informal y formal.
    \end{enumerate}
    \section{Comunicación de rutina}
    El propósito principal de la comunicación de rutina es recopilar datos, que es parte del proceso epicicloidal para cada actividad central. Usted recopila datos comunicando sus resultados y las respuestas que recibe de su audiencia deben informar los próximos pasos en su análisis de datos. Los tipos de respuestas que recibe incluyen no solo respuestas a preguntas específicas, sino también comentarios y preguntas que su audiencia tiene en respuesta a su informe. \\
    Hay tres tipos principales de comunicación informal y se clasifican en función de los objetivos que tiene para la comunicación:
    \begin{enumerate}[\bfseries 1.]
	\item  Responder a una pregunta muy enfocada, que a menudo es una pregunta técnica o una pregunta destinada a recopilar un hecho.
	\item Para ayudarlo a trabajar con algunos resultados que son desconcertantes o que no son exactamente lo que esperaba.
	\item Para obtener impresiones generales y comentarios como un medio para identificar problemas que no se le habían ocurrido para que pueda refinar su análisis de datos.
    \end{enumerate}
    Centrarse en algunos conceptos básicos le ayudará a alcanzar sus objetivos al planificar la comunicación de rutina. Estos conceptos son:
    \begin{enumerate}[\bfseries 1.]
	\item \textbf{Audiencia:} conozca a su audiencia y cuando tenga control sobre quién es la audiencia, seleccione la audiencia adecuada para el tipo de retroalimentación que está buscando.
	\item \textbf{Contenido:} Sea concentrado y conciso, pero proporcione suficiente información para que la audiencia comprenda la información que presenta y las preguntas que hace.
	\item \textbf{Estilo:} Evite la jerga. A menos que esté comunicando sobre un tema altamente técnico enfocado a una audiencia altamente técnica, es mejor usar un lenguaje y figuras y tablas que puedan ser entendidas por una audiencia más general.
	\item \textbf{Actitud:} tenga una actitud abierta y colaborativa para que esté listo para participar plenamente en un diálogo y para que su audiencia reciba el mensaje de que su objetivo no es $"$defender$"$ su pregunta o trabajo, sino más bien obtener su opinión.
    \end{enumerate}
    \section{La audiencia}
    Para muchos tipos de comunicación de rutina, tendrá la capacidad de seleccionar su audiencia, pero en algunos casos, como cuando entrega un informe provisional a su jefe o su equipo, la audiencia puede estar predeterminada. Su audiencia puede estar compuesta por otros analistas de datos, las personas que iniciaron la pregunta, su jefe y / u otros gerentes o miembros del equipo ejecutivo, analistas que no son expertos en contenido y / o alguien que represente al público en general. \\
    Si tiene una pregunta sobre cómo se recopilaron los datos de una variable en el conjunto de datos, puede dirigirse a una persona que recopiló los datos o una persona que haya trabajado con el conjunto de datos anteriormente o que fue responsable de compilar los datos. Si la pregunta es sobre el comando que se debe usar en un lenguaje de programación estadística para ejecutar un determinado tipo de prueba estadística, esta información a menudo se encuentra fácilmente mediante una búsqueda en Internet. Pero si esto falla, sería apropiado consultar a una persona que usa el lenguaje de programación en particular. Para el segundo tipo de comunicación de rutina, en la que tiene algunos resultados y no está seguro de si son los que esperaba o no son los que esperaba, probablemente se beneficiará más si involucra a más de una persona y representan una variedad de perspectivas. Las reuniones más productivas y útiles suelen incluir personas con experiencia en análisis de datos y áreas de contenido. Como regla general, cuantos más tipos de partes interesadas se comuniquen mientras realiza su proyecto de análisis de datos, mejor será su producto final. 
    \section{Contenido}
    Después de pensar un poco en sus objetivos para la comunicación, se establece en dos objetivos principales: 
    \begin{enumerate}[\bfseries 1.]
	\item Comprender si existe un mejor enfoque para manejar la no linealidad de la relación y, de ser así, cómo determinar cuál es el mejor.
	\item Para comprender más acerca de la relación no lineal que observa, incluso si esto es esperado y / o conocido y si es importante capturar la no linealidad en sus análisis. Para lograr sus objetivos, deberá proporcionar a su audiencia algo de contexto y antecedentes, pero proporcionar un trasfondo completo para el proyecto de análisis de datos y la revisión de todos los pasos que ha tomado hasta ahora es innecesario y probablemente absorberá tiempo y esfuerzo.
    \end{enumerate}
    El contenido final de su presentación, entonces, incluiría una declaración de los objetivos para la discusión, una breve descripción general del proyecto de análisis de datos, cómo el problema específico que enfrenta encaja en el proyecto general de análisis de datos, y Comunicación. 
    \section{Estilo}
    Aunque el estilo de comunicación aumenta en formalidad desde el primer al tercer tipo de comunicación de rutina, todas estas comunicaciones deben ser en gran medida informales y, excepto quizás por la comunicación enfocada sobre un pequeño problema técnico, se debe evitar la jerga. \textbf{Debido a que el propósito principal de la comunicación de rutina es obtener retroalimentación, su estilo de comunicación debe fomentar la discusión.} 
    \section{Actitud}
    Una actitud defensiva o desagradable puede sabotear todo el trabajo que ha realizado para seleccionar cuidadosamente a la audiencia, identificar cuidadosamente sus objetivos y preparar su contenido, y declarar que está buscando discusión. Su audiencia será reacia a ofrecer comentarios constructivos si sienten que sus comentarios no serán bien recibidos y usted saldrá de la reunión sin lograr sus objetivos y no estará preparado para hacer mejoras o adiciones a su análisis de datos.
\chapter{Pensamientos concluyentes}
Mientras trabaja en el desarrollo de su pregunta, explorando sus datos, modelando sus datos, interpretando sus resultados y comunicando sus resultados, recuerde siempre establecer expectativas y luego comparar el resultado de su acción con sus expectativas. Si no coinciden, identifique si el problema está en el resultado de su acción o en sus expectativas y solucione el problema para que coincidan. Si no puede identificar el problema, busque la opinión de otros y luego, cuando haya solucionado el problema, continúe con la siguiente acción. Además del marco del epiciclo, también hay actividades de análisis de datos que discutimos a lo largo del libro. Aunque todas las actividades de análisis son importantes, si tuviéramos que identificar las que son más importantes para asegurar que su análisis de datos proporcione una respuesta válida, significativa e interpretable a su pregunta, incluiríamos lo siguiente:
\begin{enumerate}[\bfseries 1.]
    \item Sea reflexivo sobre desarrollar su pregunta y utilizar la pregunta para guiarlo a lo largo de todos los pasos del análisis.
    \item Sigue el ABCs:
	\begin{enumerate}[\bfseries a)]
	    \item Siempre estar revisando.
	    \item Sea siempre desafiante.
	    \item Estar siempre comunicado.
	\end{enumerate}
\end{enumerate}
\textbf{La mejor manera para que el marco del epiciclo y estas actividades se conviertan en una segunda naturaleza es hacer mucho análisis de datos, por lo que le recomendamos que aproveche las oportunidades de análisis de datos que se le presenten. Aunque con la práctica, muchos de estos principios se convertirán en algo natural para usted, hemos descubierto que revisar estos principios nos ha ayudado a resolver una serie de problemas que enfrentamos en nuestros propios análisis.}
\end{document}

