\chapter{Introducción a la estadística y al análisis de datos}
    \setcounter{section}{2}
    \section{Medidas de localización: la media y la mediana de una muestra}

    \begin{tcolorbox}[colframe=white]
        %---------definición 1.1. media de la muestra---------
        \begin{def.}[Media de una muestra]
            Suponga que las observaciones en una muestra son $x_1 ,x_2 ,...,x_n$. La \textbf{media de la muestra} que se denota con $\overline{x}$ es 
            $$\overline{x} = \displaystyle\sum_{i=1}^n \dfrac{x_i}{n} = \dfrac{x_1 + x_2 + ... + x_n}{n}$$
            La media es simplemente un promedio numérico.\\
        \end{def.}
    \end{tcolorbox}

    \begin{tcolorbox}[colframe=white]
        %---------- definición 1.2. mediana de la muestra---------
        \begin{def.}[Mediana de una muestra]
            Dado que las observaciones en una muestra son $x_1 ,x_2 ,...,x_n$, acomodadas en \textbf{orden de magnitud creciente}, la mediana de la muestra es
            $$ \tilde{x} = \left\{ 
                \begin{tabular}{cc}
                    $x_{n+1/2}$ & si $n$ es impar\\
                    $\dfrac{1}{2} (x_{n/2} + x_{n/2+1})$ & si $n$ es par\\
                \end{tabular}
                \right. $$

            El  propósito  de  la mediana de la muestra es reflejar la tendencia central de la muestra de manera que no sea influida por los valores extremos.
        \end{def.}
    \end{tcolorbox}
    
    \section{Ejercicios}
        \begin{enumerate}[\bfseries 1.1.]
            %--------------------1.1--------------------
            \item 
        \end{enumerate}
        

    

    
