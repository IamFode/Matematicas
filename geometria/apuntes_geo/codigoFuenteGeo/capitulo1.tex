\chapter{Axiomas de incidencia y orden}

    % axioma I1
    \begin{tcolorbox}[colframe=white]
	\begin{axioma}
	   Cualquiera que sea la recta existen puntos que pertenecen a la recta y puntos que no pertenecen a la recta.
	\end{axioma}
    \end{tcolorbox}

    % axioma I2
    \begin{tcolorbox}[colframe=white]
	\begin{axioma}
	    Dados dos puntos distintos existe una única recta que los contiene.
	\end{axioma}
    \end{tcolorbox}

    % notación 1
    \begin{notacion}
	Si $P$ y $Q$ son distintos $\Rightarrow$ la recta $PQ$ es la única recta que contiene a $P$ y $Q$ ó $\overleftrightarrow{PQ}$: recta $PQ$\\
	Utilizamos letras mayúsculas para denotar puntos y minúsculas para denotar rectas.\\\\
    \end{notacion}

    \begin{proposicion}
	Dos rectas distintas o no se intersectan o se intersectan en un único punto.\\\\
	Demostración.-\; Sea $m$ y $n$ dos rectas distintas. Supongamos que $m,n$ se intersectan. Si $P, Q$ son puntos distintos donde $m,n$ se intersectan entonces por $1.2$, existe una única recta que contiene a $P$ y $Q$. Luego $m=n \; (\Rightarrow \Leftarrow)$, por tanto, $P=Q$. 
    \end{proposicion}
    
    \begin{axioma}
	Dados tres puntos de una recta, uno y sólo uno de ellos se localiza entre los dos
    \end{axioma}
