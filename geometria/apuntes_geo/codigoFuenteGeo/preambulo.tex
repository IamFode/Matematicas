\documentclass[10pt]{book}
\usepackage[text=17cm,left=2.5cm,right=2.5cm, headsep=20pt, top=2.5cm, bottom = 2cm,letterpaper,showframe = false]{geometry} %configuración página
\usepackage{latexsym,amsmath,amssymb,amsfonts} %(símbolos de la AMS).7
\parindent = 0cm  %sangria
\usepackage[T1]{fontenc} %acentos en español
\usepackage[spanish]{babel} %español capitulos y secciones
\usepackage{graphicx} %gráficos y figuras.

%---------------FORMATO de letra--------------------%

\usepackage{lmodern} % tipos de letras
\usepackage{titlesec} %formato de títulos
\usepackage[backref=page]{hyperref} %hipervinculos
\usepackage{multicol} %columnas
\usepackage{tcolorbox, empheq} %cajas
\usepackage{enumerate} %indice enumerado
\usepackage{marginnote}%notas en el margen
\tcbuselibrary{skins,breakable,listings,theorems}
\usepackage[Bjornstrup]{fncychap}%diseño de portada de capitulos
\usepackage[all]{xy}%flechas
\counterwithout{footnote}{chapter}
\usepackage{xcolor}


%------------------------------------------

\newtheorem{axioma}{\large\textbf{Axioma}}[chapter]
\newtheorem{teo}{\textbf{TEOREMA}}[chapter]%entorno para teoremas
\newtheorem{ejem}{{\textbf{EJEMPLO}}}[chapter]%entorno para ejemplos
\newtheorem{def.}{\textbf{Definición}}[chapter]%entorno para definiciones
\newtheorem{post}{\large\textbf{POSTULADO}}[chapter]%entorno de postulados
\newtheorem{col.}{\textbf{Corolario}}[chapter]
\newtheorem{ej}{\textbf{EJERCICIO}}[chapter]
\newtheorem{prop}{\textbf{Propiedad}}[chapter]
\newtheorem{lema}{\textbf{LEMA}}[chapter]
\newtheorem{prob}{\textbf{PROBLEMA}}[chapter]
\newtheorem{notacion}{\textbf{Notación}}
\newtheorem{proposicion}{\large\textbf{Proposición}}

%--------------------GRÀFICOS--------------------------

%---------------------------------

\titleformat*{\section}{\LARGE\bfseries\rmfamily}
\titleformat*{\subsection}{\Large\bfseries\rmfamily}
\titleformat*{\subsubsection}{\large\bfseries\rmfamily}
\titleformat*{\paragraph}{\normalsize\bfseries\rmfamily}
\titleformat*{\subparagraph}{\small\bfseries\rmfamily}

%------------------------------------------

\renewcommand{\labelenumi}{\Roman{enumi}.}%primer piso II) enumerate
\renewcommand{\labelenumii}{\arabic{enumii}$)$}%segundo piso 2)
\renewcommand{\labelenumiii}{\alph{enumiii}$)$}%tercer piso a)
\renewcommand{\labelenumiv}{$\bullet$}%cuarto piso (punto)

%----------Formato título de capítulos-------------

\usepackage{titlesec}
\renewcommand{\thechapter}{\arabic{chapter}}
\titleformat{\chapter}[display]
{\titlerule[2pt]
\vspace{4ex}\bfseries\sffamily\huge}
{\filleft\Huge\thechapter}
{2ex}
{\filleft}

\usepackage[htt]{hyphenat}
