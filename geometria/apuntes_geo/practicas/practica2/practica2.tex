\documentclass[10pt]{article}				
\usepackage[text=17cm,left=2.5cm,right=2.5cm, headsep=20pt, top=2.5cm, bottom = 2cm,letterpaper,showframe = false]{geometry} %configuración página
\usepackage{latexsym,amsmath,amssymb,amsfonts} %(símbolos de la AMS).7
\parindent = 0cm  %sangria
\usepackage[T1]{fontenc} %acentos en español
\usepackage[spanish]{babel} %español capitulos y secciones
\usepackage{graphicx} %gráficos y figuras.
%-----------------------------------------%
\usepackage{multicol}
\usepackage{titlesec}
\usepackage[rflt]{floatflt}
\usepackage{wrapfig} 
\usepackage{tikz}\usetikzlibrary{shapes.misc}
\usepackage{tikz,tkz-tab}						
\usetikzlibrary{matrix,arrows, positioning,shadows,shadings,backgrounds,
calc, shapes, tikzmark}
\usepackage{tcolorbox, empheq}
\tcbuselibrary{skins,breakable,listings,theorems}
\usepackage{xparse}							
\usepackage{pstricks}							
\usepackage[Bjornstrup]{fncychap}			
\usepackage{rotating}
\usepackage{enumerate}
\usepackage{booktabs}
\usepackage{synttree} 
\usepackage{chngcntr}
\usepackage{venndiagram}
\usepackage[all]{xy}
\usepackage{xcolor}
\usepackage{tikz}
\usetikzlibrary{datavisualization.formats.functions}
\usepackage{marginnote}										
\usepackage{fancyhdr}

%------------------------------------------
\renewcommand{\labelenumi}{\Roman{enumi}.}		%primer piso II) enumerate
\renewcommand{\labelenumii}{\arabic{enumii}$)$ }%segundo piso 2)
\renewcommand{\labelenumiii}{\alph{enumiii}$)$ }%tercer piso a)
\renewcommand{\labelenumiv}{$\bullet$}		

\pagestyle{fancy}
\fancyhead[R]{Geometría I}
\fancyhead[L]{Práctica III}

\begin{document}
\begin{tabular}{r l }
Universidad: & \textbf{Mayor de San Ándres.}\\
Asignatura: & \textbf{Geometría I.}\\
 Práctica: & II.\\ 
Alumno: & \textbf{PAREDES AGUILERA CHRISTIAN LIMBERT.}
\end{tabular}
\begin{flushleft}
\begin{tikzpicture}
\draw(0,1)--(16.5,1);
\end{tikzpicture}
\end{flushleft}


    \begin{enumerate}[\large \bfseries 1.]

	%--------------------1.
	\item Sean $A,B$ y $C$ puntos de una recta. Haga un diseño representándolos, sabiendo que $\overline{AB}=3$, $\overline{AC}=2$ y $\overline{BC}=5$\\\\
	Respuesta.-\;
    \begin{center}
	\begin{tikzpicture}
	    \draw[<->](-2,0)--(7,0);
	    \filldraw[black](3,0) circle(2pt) node[below]{$A$};
	    \filldraw[black](0,0) circle(2pt) node[below]{$B$};
	    \filldraw[black](5,0) circle(2pt) node[below]{$C$};
	\end{tikzpicture}
    \end{center}

	%--------------------2.
	\item Repita el ejercicio anterior, sabiendo que $C$ está entre $A$ y $B$ y que $\overline{AB}=7$ y $\overline{AC}=5.$\\\\
	Respuesta.-\;
    \begin{center}
	\begin{tikzpicture}
	    \draw[<->](-2,0)--(9,0);
	    \filldraw[black](0,0) circle(2pt) node[below]{$A$};
	    \filldraw[black](7,0) circle(2pt) node[below]{$B$};
	    \filldraw[black](5,0) circle(2pt) node[below]{$C$};
	\end{tikzpicture}
    \end{center}

	%--------------------3.
	\item Diseñe una recta y sobre ella marque dos puntos $A$ y $B$. Suponga que la coordenada del punto $A$ sea cero y la del punto $B$ sea $1$. Marque ahora puntos cuyas coordenadas son $3,5,5/2,1/3,3/2,2,-1,-2,-5,-1/3,-5/3$\\\\
	Respuesta.-\;
    \begin{center}
	\begin{tikzpicture}[scale=1.3]
	    \draw[<->](-6,0)--(6,0);
	    \filldraw[red](0,0) circle(1.5pt) node[below]{$A$} node[above]{$0$};
	    \filldraw[red](1,0) circle(1.5pt) node[below]{$B$} node[above]{$1$};
	    \filldraw[black](3,0) circle(1pt) node[below]{$p_1$} node[above]{\tiny$3$};
	    \filldraw[black](5,0) circle(1pt) node[below]{$p_2$} node[above]{\tiny$5$};
	    \filldraw[black](5/2,0) circle(1pt) node[below]{$p_3$} node[above]{\tiny$5/2$};
	    \filldraw[black](1/3,0) circle(1pt) node[below]{$p_4$} node[above]{\tiny$1/3$};
	    \filldraw[black](3/2,0) circle(1pt) node[below]{$p_5$} node[above]{\tiny$3/2$};
	    \filldraw[black](2,0) circle(1pt) node[below]{$p_6$} node[above]{\tiny$2$};
	    \filldraw[black](-1,0) circle(1pt) node[below]{$p_7$} node[above]{\tiny$-1$};
	    \filldraw[black](-2,0) circle(1pt) node[below]{$p_8$} node[above]{\tiny$-2$};
	    \filldraw[black](-5,0) circle(1pt) node[below]{$p_9$} node[above]{\tiny$-5$};
	    \filldraw[black](-1/3,0) circle(1pt) node[below]{$p_{10}$} node[above]{\tiny$-1/3$};
	    \filldraw[black](-5/3,0) circle(1pt) node[below]{$p_{11}$}  node[above]{\tiny$-5/2$};
	\end{tikzpicture}
    \end{center}

	%--------------------4.
	\item Sean $A_1$ y $A_2$ puntos de coordenadas $1$ y $2$. De la coordenada del punto medio $A_3$ del segmento $A_1 A_2$. De la coordenada del punto medio $A_4$ del segmento $A_2A_3$. De la coordenada del punto medio $A_5$ del segmento $A_3A_4$.\\\\
	Respuesta.-\; Dado que $A_3$ es el punto medio del segmento $A_1 A_2$, la coordenada $A_3$ será la media aritmética.
	$$A_3=\dfrac{A_1 + A_2}{2}=\dfrac{1+2}{2}=\dfrac{3}{2}$$
	Luego calculamos análogamente para los otros puntos.
	$$A_4=\dfrac{\dfrac{3}{2}+2}{2}=\dfrac{7}{4}$$
	$$A_5=\dfrac{\dfrac{3}{2}+\dfrac{7}{4}}{2}=\dfrac{13}{8}$$\\\\

	%--------------------5.
	\item Sean $a,b,c,d$ números reales distintos de cero. Pruebe que, si $\dfrac{a}{b}=\dfrac{c}{d}$ entonces
	\begin{enumerate}[\bfseries a)]
	    %----------a)
	    \item $\dfrac{a}{c}=\dfrac{b}{d}$ y $\dfrac{d}{b}=\dfrac{c}{a}$\\\\
	    Demostración.-\; Sea $\dfrac{a}{b}=\dfrac{c}{d}$ entonces por hipótesis $\dfrac{a}{b}\cdot \dfrac{b}{c}=\dfrac{c}{d} \cdot \dfrac{b}{c}$ luego $\dfrac{a}{c}=\dfrac{b}{d}$\\\\ 

	    %----------b)
	    \item $\dfrac{a+c}{a}=\dfrac{c+d}{c}$ y $\dfrac{a-c}{a}=\dfrac{c-d}{c}$\\\\
	    Demostración.-\; Sea $\dfrac{a}{b}=\dfrac{c}{d}$ entonces $\dfrac{db}{ac}\cdot \dfrac{a}{b}$ luego $1+\dfrac{d}{c}=1+\dfrac{b}{a}$, de donde $\dfrac{c}{c}\cdot \dfrac{d}{c}=\dfrac{a}{a}\cdot \dfrac{b}{a}$ y por lo tanto $\dfrac{c+d}{c}=\dfrac{b+a}{a}$\\\\

	    %----------c)
	    \item $\dfrac{a+b}{b}=\dfrac{c+d}{d}$ y $\dfrac{a-b}{b}=\dfrac{c-d}{d}$\\\\
	    Demostración.-\; Sea $\dfrac{a}{b}=\dfrac{c}{d}$ entonces $\dfrac{a}{b}\cdot \dfrac{bd}{ac}=\dfrac{c}{d}\cdot \dfrac{db}{ac}$ luego similar al ejercicio anterior tenemos $-1 \cdot \dfrac{d}{c}=-1 \cdot \dfrac{b}{a}$ y por lo tanto $\dfrac{c-d}{c}=\dfrac{a-b}{a}$\\\\

	\end{enumerate}

	%-------------------6.
	\item Si $P$ es el punto de intersección de los círculos de radio $r$ y centros en $A$ y $B$, muestre que $\overline{PA}=\overline{PB}$\\\\
	\begin{center}
	    \begin{tikzpicture}
		\draw(0,0) circle(1cm);
		\draw(2,0) circle(1cm);
		\filldraw[black](1,0) circle(1pt) node[below left]{\tiny$P$};
		\filldraw[black](0,0) circle(1pt) node[left]{\tiny$A$};
		\filldraw[black](2,0) circle(1pt) node[ right]{\tiny$B$};
		\draw(0,0)--(1,0)node[above left ]{\tiny$\overline{PA}$};
		\draw(1,0)--(2,0)node[above left]{\tiny$\overline{PB}$};
	    \end{tikzpicture}
	\end{center}
	Demostración.-\; Como el punto $P$ está en la intersección de los dos círculos. Entonces $P$ pertenece al círculo con centro $A$ y radio $r$, y por definición de círculo, $PA = r$, igualmente $P$ pertenece al círculo con centro $B$ y radio $r$, por definición de círculo, $PB = r$, lo que implica que $PA = PB$.\\\\

	%--------------------7.
	\item Usando regla y compás (regla no numerada), describa un método para la construcción de un triángulo con dos lados de misma longitud. (Un triángulo así, es llamado triángulo isósceles).\\\\
	Respuesta.-\; Considere un segmento $AB$. Con un compás centrada en $A$, dibuja una circunferencia de radio $AB$. Ahora con el centro en $B$, dibuja un círculo de radio $BA$. La intersección entre los dos círculos generará los puntos $C$ y $D$. Haciendo el triángulo $CAD$ tendremos un triángulo isósceles con base $CD$ y lados $CA,AD$.
	\begin{center}
	    \begin{tikzpicture}
		\draw[gray](0,0)--(1,0);
		\draw[red](0,0)--(0.5,.85);
		\draw[red](0,0)--(.5,-.85);
		\draw[red](.5,-.85)--(.5,.85);
		\draw(0,0) circle(1cm);
		\filldraw[black](0,0) circle(1pt) node[below left]{\tiny$A$};
		\draw(1,0) circle(1cm);
		\filldraw[black](1,0) circle(1pt) node[below right]{\tiny$B$};
		\filldraw[black](0.5,.85) circle(1pt) node[above]{\tiny$D$};
		\filldraw[black](0.5,-.85) circle(1pt) node[below]{\tiny$C$};

	    \end{tikzpicture}
	\end{center}

	%--------------------8.
	\item Describa un método para construir un triángulo con sus tres lado de misma longitud.( Un triángulo así, es llamado triángulo equilátero ).\\\\
	Respuesta.-\; Se traza una recta y se marca dos puntos $A$ y $B$ en ella. Con el centro en $A$ y luego en $B$, se generan dos circunferencias de radio $r$ generando el punto $C$, después trazamos los segmentos $AC, AB$ y $BC$ que generará $\triangle ABC$ con lados iguales a $r$.
	\begin{center}
	    \begin{tikzpicture}
		\draw[red](0,0)--(1,0);
		\draw[red](0,0)--(0.5,.85);
		\draw[red](1,0)--(.5,.85);
		\draw(0,0) circle(1cm);
		\filldraw[black](0,0) circle(1pt) node[below left]{\tiny$A$};
		\draw(1,0) circle(1cm);
		\filldraw[black](1,0) circle(1pt) node[below right]{\tiny$B$};
		\filldraw[black](0.5,.85) circle(1pt) node[above]{\tiny$C$};
	    \end{tikzpicture}
	\end{center}

	%--------------------9.
	\item Muestre que, si $a<b$ entonces $a<\dfrac{a+b}{2}$ y $b>\dfrac{a+b}{2}$\\\\
	Demostración.-\; Daremos una demostración que va mas allá del ejercicio en si planteado para entender de mejor manera esta proposición.\\\\
	Demostrar que si $0<a < b$, entonces $$a<\sqrt{ab}<\dfrac{a+b}{2} < b$$\\
	\begin{enumerate}[1.]
	    \item $a<\sqrt{ab}$\\\\
Si \; $4a<b$ entonces $a^2<ab$ y por raíz cuadrada dado que $a,\;b>0$ entonces $a<\sqrt{ab}$\\
	    \item $\sqrt{ab}<\dfrac{a+b}{2}$\\\\
En vista de que $a, \; b > 0$ y $a<b$ entonces $a-b>0$, \; $(a-b)^2>0$ por lo tanto, $a^2-2ab+b^2>0 \Rightarrow 2ab< a^2+b^2 \Rightarrow 2ab-2ab+2ab<a^2+b^2 \Rightarrow 4ab < a^2+2ab +b^2 \Rightarrow 4ab < (a+b^)2 \Rightarrow ab < \displaystyle \left( \frac{a+b}{2} \right) ^2 \Rightarrow \sqrt{ab}<\frac{a+b}{2} $ \\
	    \item $\displaystyle\frac{a+b}{2}<b$\\\\
Si $a<b$ entonces $a+b<2b$ por lo tanto $\displaystyle\frac{a+b}{2}<b$\\\\
	\end{enumerate}

	%--------------------10.
	\item ¿Es posible construir un triángulo con lados de longitud $3,8$ y $5$?\\\\
	Respuesta.-\; No, ya que la desigualdad triangular establece que la suma de dos lados cualesquiera de un triángulo es mayor que el tercer lado, es decir, si tomamos las medidas de los lados $5$ y $3$, tendremos $8$ que será igual a la tercera medida.\\\\

	%--------------------11.
	\item El círculo de radio $r_1$ centrado en $A$ intersecta al círculo de radio $r_2$ centrado en $B$ en exactamente dos puntos. ¿Qué se puede afirmar sobre $\overline{AB}$?\\\\
	Respuesta.-\; Considere el círculo de radio $r_2$ con centro en $A$ y el círculo de radio $r_1$ con centro en $B$ y cuyo segmento $AB$ forma los puntos $C$ y $D$. Note que $AB = AD + CB - CD$ y también que $AD = r_2, CB = r_1$ y que $CD$ es un segmento no nulo. Luego observe que  $AB = r_2 + r_1 - CD$, lo que implica que $AB <r_2 + r_1$\\
	\begin{center}
	    \begin{tikzpicture}
		\draw[red](-.25,0)--(1.25,0);
		\draw[red](-.25,0)--(-.5,1);
		\draw[red](1.25,0)--(1.6,-1);
		\draw(-.25,0) circle(1cm);
		\draw(1.25,0) circle(1cm);
		\filldraw[black](-.25,0) circle(1pt) node[below left]{\tiny$A$};
		\filldraw[black](1.25,0) circle(1pt) node[below right]{\tiny$B$};
		\filldraw[black](0.25,0) circle(1pt) node[above left]{\tiny$C$};
		\filldraw[black](0.75,0) circle(1pt) node[above right]{\tiny$D$};
		\filldraw[black](1.6,-1) circle(1pt) node[below]{\tiny$r_1$};
		\filldraw[black](-.5,1) circle(1pt) node[above]{\tiny$r_2$};
	    \end{tikzpicture}
	\end{center}

	%--------------------12.
	\item Considere un círculo de radio $r$ y centro $A$. Sean $B$ y $C$ puntos de este círculo. ¿Qué se puede afirmar sobre el triángulo $ABC$?\\\\
	\begin{center}
	    \begin{tikzpicture}
		\draw(0,0)--(.7,.7)node[above]{\tiny$r$};
		\draw[red](0,0)--(-.7,-.7);
		\draw[red](0,0)--(.7,-.7);
		\draw[red](-.7,-.7)--(.7,-.7);
		\draw(0,0) circle(1cm);
		\filldraw[black](0,0) circle(1pt) node[above]{\tiny$A$};
		\filldraw[black](-.7,-.7) circle(1pt) node[below]{\tiny$B$};
		\filldraw[black](.7,-.7) circle(1pt) node[below]{\tiny$C$};
	    \end{tikzpicture}
	\end{center}
	Respuesta.-\; Si los puntos $B$ y $C$ pertenecen a la circunferencia que forma el círculo entonces $AB = AC = r$ por lo tanto el triángulo es isósceles con base $AB$.\\\\

	%--------------------13.
	\item Considere un círculo de radio $r$ y centro $O$. Sea $A$ un punto de este círculo y sea $B$ un punto tal que el triángulo $OAB$ es equilátero. ¿Cuál es la posición del punto $B$ en relación al círculo?\\\\
	Respuesta.-\; Dado que el triángulo es equilátero y uno de sus lados es el segmento $OA$ de tamaño $r$, entonces $OB = r$, luego el punto $B$ está a una distancia $r$ del centro del círculo, es decir, $B$ pertenece a la circunferencia.\\\\

	%--------------------14.
	\item Dos círculos de mismo radio y centro $A$ y $B$ se intersectan en dos puntos $C$ y $D$. ¿Qué se puede afirmar sobre los triángulos $ABC$ y $ACD$? ¿Y, sobre el cuadrilátero $ACBD$?\\\\
	Respuesta.-\; 
	\begin{center}
	    \begin{tikzpicture}
		\draw[red](0,0)--(.5,.85)--(1,0)--(.5,-.85)--(0,0)--(1,0);
		\draw[red](.5,.85)--(.5,-.85);
		\draw(0,0) circle(1cm);
		\draw(1,0) circle(1cm);
		\filldraw[black](0,0) circle(1pt) node[below left]{\tiny$A$};
		\filldraw[black](1,0) circle(1pt) node[below right]{\tiny$B$};
		\filldraw[black](.5,.85) circle(1pt) node[above]{\tiny$C$};
		\filldraw[black](.5,-.85) circle(1pt) node[below]{\tiny$D$};
	    \end{tikzpicture}
	\end{center}
	Los triángulos $ABC$ y $ACD$ son isósceles porque $AC, BC = r $ y $AD = r$  así también $BD = r$. Como el paralelogramo ACBD está formado por la unión de $\triangle ABC$ y $\triangle ADB$, sus lados serían los segmentos que forman el triángulo,  luego $AC = BC = AD = BD = r$. Entonces el polígono es un cuadrilátero de lados iguales y los triángulos son isósceles.\\\\

	%--------------------15.
	\item Dado un segmento $AB$ muestre que existe y es único, un punto $C$ entre $A$ y $B$ tal que $$\dfrac{\overline{AC}}{\overline{BC}}=a,$$ donde $a$ es cualquier número real positivo.\\\\
	Demostración.-\; Sean $x, b$ y $c$ las coordenadas de los puntos $A, B$ y $C$. Podemos suponer que $a <b <c$. Luego para el caso de $x> b> c$, se resuelve de forma totalmente análoga. Entonces, por el axioma $III_2$ el problema a demostrar pasa por la existencia de un solo punto $B$ entre $A$ y $C$ tal que $\dfrac{m(AC)}{m(BC)}=a$, equivale a mostrar que existe un único número real $b$ tal que $x<b<c$ y $\dfrac{b-x}{c-b}=a$.\\
	Resolviendo en $b$ obtenemos que la única solución es $b=\dfrac{x+ca}{1+a}$. Finalmente, queda demostrar que este $b$ encontrado satisfaga a $x<b<c$. Es decir,
	$$x=\dfrac{x+ax}{1+a}<\dfrac{x+ca}{1+a}<\dfrac{c+ca}{1+a}$$
	La unicidad del punto $B$ también es una consecuencia del axioma $III_2$.\\\\

	%--------------------16.
	\item Pruebe que un segmento de recta que une un punto del círculo, con un punto dentro del mismo, tiene un punto en común con el círculo.\\\\
	Demostración.-\; Sea $C$ cualquier punto fuera de un círculo de centro $O$, entonces $OC> r$ donde $r$ es el radio del círculo. Entonces, hay un punto $D \in OC$ tal que $\overline{OD} = r$. Dado que el círculo está formado por todos los puntos en el plano que están a una distancia $r$ del punto $O$, entonces el punto $D$ pertenece a la intersección del segmento $OC$ con la circunferencia.\\\\

	%--------------------17.
	\item Dados los puntos $A$ y $B$ y un número real $r$ mayor que $\overline{AB}$, el conjunto de los puntos $C$ que satisfacen $\overline{CA}+\overline{CB}=r$ es llamado elipse. Establezca los conceptos de región interior y de región exterior a un elipse.\\\\
	Respuesta.-\; Si $\overline{CA} + \overline{CB}> r$, entonces el conjunto de puntos es externo. Si $\overline{CA} + \overline{CB} <r$, entonces el conjunto de puntos será interno.\\\\

	%--------------------18.
	\item Un conjunto $M$ de puntos del plano es acotado si existe un círculo $C$ tal que todos los puntos de $M$ están dentro de $C$. Pruebe que cualquier conjunto finito de puntos es acotado. Pruebe también que los segmentos son acotados. Concluya el mismo resultado para los triángulos.\\\\
	Demostración.-\; Dado el conjunto de puntos $P_1, P_2, ..., P_n$ tome un solo punto $P_i$ que usaremos para el centro de la circunferencia, para cada punto $P_j$ con $i\neq j$ y $j$ variando de $1$ a $n$ quitando el propio $i$, pasará a un segmento diferente. Deje que $P_i P_j$ sea el más grande de todos los segmentos, por lo que se marca a sí mismo un punto $Q (P_1 - P_j - Q)$ en la línea que pasa por el segmento de modo que por $P_1 Q$ definimos un círculo de radio $r = P_1 Q$ que contendrá todos los demás, ya que el segmento que establece su radio con relación al centro $P_1$ es mayor que los demás definidos por todos los demás puntos.\\\\

	%--------------------19.
	\item Pruebe que la unión de una cantidad finita de conjuntos acotados es también un conjunto acotado.\\\\
	Demostración.-\;

	%--------------------20.
	\item Muestre que dado un punto $P$ y un conjunto acotado $M$, existe un círculo $C$ con centro en $P$ tal que todos los puntos de $M$ están dentro de $C.$\\\\
	Demostración.-\;

	%--------------------21.
	\item Pruebe que las rectas son conjuntos no acotados.\\\\
	Demostración.-\;



    \end{enumerate}


\end{document}
