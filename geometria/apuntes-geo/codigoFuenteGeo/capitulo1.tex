\chapter{}
\begin{proposicion} Veamos la siguiente proposición

    \begin{enumerate}[\bfseries a)]
	\item $S_{AB} \cup S_{BA}$ es la recta $AB$
	\item $S_{AB} \cap S_{BA} =$ segmento $AB$\\
    \end{enumerate}
    Demostración.- \; Sea $m$ la recta $AB$. Es claro que $S_{AB} \subset m$, $S_{BA} \subset m$ luego, $S_{AB} \cup \S_{BA} \subset m$\\
    Por otro lado, si $C \in m$ entonces por el axioma $II_1$ existen 3 posibilidades:
    \begin{enumerate}[\bfseries 1)]
    \item $C$ está entre $A$ y $B$.
    \item $A$ está entre $C$ y $B$.
    \item $B$ está entre $A$ y $C$.
    \end{enumerate}
    Si pasa $1)$, entonces $C \in \overline{AB}$ y como $\overline{AB} \subset S_{AB},$ luego $C \in S_{AB}.$ Por tanto $C \in S_{AB} \cup S_{BA}.$\\
    Si pasa $2) \Rightarrow C \in \in S_{BC} \Rightarrow C \in S_{AB} \cup S_{Ba}.$ \\
    Si pasa $3) \Rightarrow C \in S_{AB} \Rightarrow C \in S_{A B} S_{B}.$\\\\
    Por tanto, $m \subset S_{AB} S_{BA}$. En conclusión de $1)$ y $2)$, $m = S_{AB} \cup S_{BA}$\\\\ 

\end{proposicion}

\begin{ej}
    Demostrar que si $A-B-C \Rightarrow \overleftrightarrow{AB} = S_{BA} \cup S_{BC}$ 
\end{ej}.:
